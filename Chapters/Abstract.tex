% Structure
% =========
%
% General
% ———————
%
% - A statement of your main topic, purpose and objectives
% - A brief description of the methodology
% - An overview of the most significant findings or arguments
% - A summary of your conclusions and recommendations
% - Reflect structure of thesis
%
% Thesis
% ——————
%
% - Topic: Parsing, Configuration Data (YAML)
% - Purpose/Objective: Find ideal parsing technique for configuration data
% - Findings:
%   - ANTLR provides best mix of ease of use, error reporting and descent
%     performance
%
% Overview
% ========
%
% Problem
% ———————
%
% - A lot of different configuration file formats
% - Need to parse these formats
% - Parsing generators and libraries exist, but
%
%   - libraries have problems with comment and whitespace handling
%   - handwritten parsers often do not offer extended features, such as
%     Unicode handling and error recovery
%
% .
%
% References
% ==========
%
% - https://patthomson.net/2013/12/11/writing-the-thesis-abstract
% - https://www.sfu.ca/~jcnesbit/HowToWriteAbstract.htm
% - https://www.scribbr.com/dissertation/abstract

\begin{abstract}
Parsing – determining structure of input according to a language grammar – has been an important area of research in computer science for decades. One might assume that the results of this research, such as parser generators and parser libraries are prevalent tools, when a programmer decides to work with any computer language. However, this is not the case. Especially for custom and small languages, such as configuration file formats, people tend to write their own parsing code often ignoring lessons learned in the past. While this might work reasonably well for regular languages, there are many features parser generators and parsing libraries offer that developers will probably not implement in their custom parsers, such as Unicode support and error recovery.
\end{abstract}
