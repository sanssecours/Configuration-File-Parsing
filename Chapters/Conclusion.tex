\chapter{Conclusion \& Future Work}
\label{sec:conclusion_and_future_work}

\section{Conclusion}

In the thesis we compared different parsing techniques using the \nameref{sec:keyset} data structure of the configuration framework \href{https://www.libelektra.org}{Elektra}. The aim of this work was to find the most promising parsing technique for configuration files using a subset of the language \glstext{YAML} as example.

In a \hyperref[sec:evaluation]{detailed evaluation} we determined the answer to our auxiliary research questions.

\speed*

The \hyperref[sec:run_time_performance]{benchmarks} showed a big difference between the runtime of the \hyperref[sec:design_challenges_and_decisions]{parsing plugins} especially for large files. However, at least for our example data, all the plugins \hyperref[fig:benchmark_results_generated_above_1000]{showed linear runtime behavior}. Even the PEG parser library \gls{PEGTL} that has a theoretical exponential runtime in the worst case showed this linear behavior. Interestingly, \gls{YAEP}, the library that uses one of the most powerful parsing techniques tested (Earley parsing), showed the \hyperref[fig:benchmark_generated]{best runtime performance on macOS}. On Linux the library was only \hyperref[fig:benchmark_generated]{slightly slower than the fastest parser, based on Bisons’ LALR code}.

\memory*

All of the parser plugins showed a linear peak memory usage increase for a linear increase of the size of the input. The difference between the peak memory usage was substantial, though. The \glstext{YAEP} parser needed the least amount of memory, while the Bison parser required about 20\% more memory. The other parsers required about three, up to more than four times more peak heap memory.

\size*

\closeness*

We showed that the \gls{PEG} library \gls{PEGTL} allowed us to stay much closer to the representation of the \href{http://yaml.org/spec/1.2/spec.html}{specification of our example language YAML}. This closeness provides utility, when we compare the ease of extensibility of the language grammar. However, in the case of YAML, the language specification is rather low-level. This means the extension of the support code that converts the data in Elektra’s data structures takes considerable more effort, than for the lexer based parsers.

The answer to the research question above helped us to answer our main research question.

\overview*

In the end, \gls{ANTLR}, the parsing engine

\begin{itemize}
  \item that provides the most complete support code,
  \item that produces good error messages, and
  \item offers error recovery without any changes to the grammar
\end{itemize}

showed the best overall results according to our \hyperref[sec:evaluation]{evaluation}. Bison and \gls{YAEP} also showed promising results, while yaml-cpp and PEGTL did not fit all \hyperref[sec:requirements_extended_yaml_plugin]{requirements for an extended YAML storage plugin well}. In the beginning of the thesis we also considered using the bidirectional programming library Augeas and the parser combinator library mpc. However, in the implementation phase we found that both of these libraries are unsatisfactory for our needs. Augeas is not able to process the context sensitive language YAML, and mpc does not seem to offer any significant advantage over the similar library \gls{PEGTL}.

Overall this thesis contributes a thorough comparison of state of the art implementations of parsing techniques in the context of configuration data. While the current literature mostly compares different parsers that produce different output, we verify that our parsers produce the same data, providing a fair comparison of the given parser engines. Unlike other research we do not only compare the execution time and memory usage of the parser plugins, but also provide a detailed analysis of other important criteria, such as the error handling capabilities of the evaluated parsers.

\section{Future Work}

While we think that the comparison presented in this theses is thorough, we found some limitations future research could take into consideration.

\subsection{Additional Data Formats}

The \glstext{YAML} file format we used as example is quite complicated. We therefore used a custom lexer instead of a standard tool, like ANTLR’s lexer or flex, to make the parsing process of the white space rules of the language easier. It would make sense to also write, generate, and compare parsing code for simpler data configuration formats, such as \gls{JSON}, TOML or INI. These formats should make it easier to use a standard lexing tool, and allow us to determine how much influence a lexing tool has on the overall parsing process.

\subsection{Type Support}

We did not consider proper type support in the thesis. While we added support for binary data to the \LinkYAMLCPP{} plugin, most of the other code we wrote does not support types properly. This can lead to problems, such as \href{https://issues.libelektra.org/2833}{unwanted conversions from boolean data to integer values}.

\subsection{Lexer Level Error Messages}

The custom lexer code written in this thesis does not detect or report errors at the \gls{token} level. Adding support for this feature should be relatively easy and allow us to compare error messages for common low-level mistakes. This is especially interesting, since we would be able to assess, if the PEG library \gls{PEGTL} is able to provide the same error message quality as handwritten custom code for low-level errors.

\subsection{Additional Parser Engines/Generators}

To improve the comparability of the runtime and memory benchmarks we only considered tools written in C or C++ in this thesis. However, some of the most interesting parsing research focuses on tools written in other programming languages.

\begin{description}[style=multiline, leftmargin=3cm, font=\bfseries]
  \item[LPegLabel (Lua)] This library supports some of the recent interesting features for error handling in \glspl{PEG}, such as labeled failures~\cite{maidl2016labeled} and syntax error recovery~\cite{medeiros2018recovery}.

  \item[Marpa (Perl)] Marpa is a parsing library based on Earley’s parsing algorithm. The library implements improvements to the algorithm from \citeauthor{leo1991general}~\cite{leo1991general} and \citeauthor{aycock2002practical}~\cite{aycock2002practical}. According to the author~\cite{kegler2019marpa}: “Marpa is intended to replace, and to go well beyond, recursive descent and the yacc family of parsers”.

  \item [Menhir (OCaml)] Menhir is a \glstext{LR} parser generator that provides support for “example based error reporting”~\cite{jeffery2003generating, kaestner2018compcert, pottier2019menhir}. In theory we should be able to generate parsers with Menhir that produce error messages comparable to the ones of handwritten recursive descent parsers, used in tools such as Clang or GCC~\cite[p. 2]{kaestner2018compcert}.
\end{description}
