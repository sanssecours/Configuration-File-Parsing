\begin{kurzfassung}
Parsen – das Erfassen der Struktur eines Textes an Hand einer gegebenen Grammatik – ist seit Jahrzehnten ein wichtiger Teil der Forschung in der Informatik. Man würde vermuten dass die Resultate dieser Forschung, Parsergeneratoren und Parserbibliotheken, vorherrschende Werkzeuge sind, wenn Programmierer sich entscheiden eine Computersprache zu verarbeiten. Das ist jedoch nicht der Fall. Insbesondere für spezialisierte und einfache Sprachen, wie sie z.B. für Konfigurationsdaten verwendet werden, entscheiden sich Programmierer oft ihren eigenen Parsercode zu schreiben und ignorieren dabei in der Vergangenheit gewonnene Erkenntnisse. Während dieser Ansatz für reguläre Sprachen gute Ergebnisse liefern kann, gibt es viele Funktionen die Parsergeneratoren und Parserbibliotheken implementieren, die ein Entwickler üblicherweise nicht in seinen handgeschriebenen Parsercode integrieren wird, wie z.B. Unicode-Support und Fehlerbehebung. Die Verwendung einer Bibliothek für eine bestimmte Computersprache kann einige dieser Problemen lösen. Allerdings existieren solche Bibliotheken nicht für jede Sprache oder bieten keinen Zugriff auf wichtige Daten eines Textes, wie z.B. Kommentare. Um die gegenwärtige Situation zu verbessern beschäftigt sich diese Diplomarbeit mit aktuellen Parserwerkzeugen und vergleicht diese. Für diese Zweck parsen wir eine Teilmenge der Sprache \glstext{YAML} und wandeln Dokumente in dieser Sprache in die Datenstrukturen des Konfiguration-Frameworks Elektra um. Diese gemeinsame Datenstruktur erlaubt einen sachlichen Vergleich der Parserwerkzeuge die üblicherweise verschiedene Arten von Syntax-Bäumen erzeugen oder benutzerdefinierten Code während des Parse-Vorgangs ausführen.

Diese Diplomarbeit beginnt mit einem Literaturrecherche, die den aktuellen Stand der Forschung im Bereich Parsing zeigt. Nach dieser Recherche wählen wir Parserwerkzeuge für die vielversprechendsten Pasertechniken, \glstext{ALL(*)}, LR, Earley Parser, \glstext{PEG}, Paserkombinatoren und bidirektionale Programmierung. In einer Diskussion mit einigen der Elektra-Entwickler entschieden wir uns dann für die Funktionen, die unsere \glstext{YAML}-Teilmenge unterstützen soll. Schließlich gehen wir auf die Abbildung zwischen YAML und der Datenstrukturen von Elektra ein, und implementieren die Parser für die \glstext{YAML}-Teilmenge. In der Implementierungsphase entschieden wir uns dabei gegen eine Implementierung mit der Paserkombinatorbibliothek mpc und der Bibliothek für bidirektionale Programmierung Augeas. Während Augeas nicht mächtig genug ist um die kontextsensitive Sprache \glstext{YAML} zu verarbeiten, konnten wir keinen Vorteil von mpc gegenüber der konzeptuell ähnlichen \glstext{PEG}-Bibliothek \glstext{PEGTL} finden.

Die Messungen und Vergleiche der verschiedenen Funktionen der Parser-Werkzeuge zeigen, dass für unser Beispielsprache \glstext{ANTLR} den größten Funktionsumfang, und gute Fehlermeldungen, ohne Änderungen an der Grammatik zu benötigen, liefert. Während der Code des Parsergenerators weder die schnellste Laufzeit, noch den geringsten Speicherverbrauch aufweist, zeigen unsere Messungen des Codes doch eine angemessene Leistung.
\end{kurzfassung}
