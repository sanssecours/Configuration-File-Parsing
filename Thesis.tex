%!TEX TS-program = xelatex

\documentclass[draft, oneside, final, x11names, svgnames]{Template/vutinfth}

% -- Packages ------------------------------------------------------------------

\usepackage{bchart}
\usepackage[sorting=ynt, style=alphabetic, backend=biber]{biblatex}
\usepackage{booktabs}
\usepackage{enumitem}
\usepackage{fontspec}
\usepackage{float}
\usepackage{microtype}
\usepackage{minted}
\usepackage{morewrites}
\usepackage{nag}
\usepackage{placeins}
\usepackage{subcaption}
\usepackage{tcolorbox}
\usepackage{thmtools}
\usepackage{titlesec}
\usepackage{unicode-math}
\usepackage{xcolor}
\usepackage{xelatexemoji}
\usepackage{hyperref}
\usepackage{cleveref}
\usepackage[acronym, toc]{glossaries}

% -- Attributes ----------------------------------------------------------------

\newcommand{\authorname}{René Schwaiger}
\newcommand{\thesistitle}{Parsing of Configuration Files}

\setauthor{}{\authorname}{}{male}
\setadvisor{Ao.Univ.Prof. Dipl.-Ing. Dr.techn.}{Franz Puntigam}{}{male}
\setfirstassistant{Senior Lecturer Dipl.-Ing. Dr.techn.}{Markus Raab}{BSc.}{male}

\setaddress{Waldmüllergasse 9}
\setregnumber{0425176}
\setdate{\day}{\month}{\year}
\settitle{\thesistitle}{Parsing von Konfigurationsdateien}

\setthesis{master}
\setmasterdegree{dipl.}

\setcurriculum{Computer Engineering}{Technische Informatik}

% -- Bibliography --------------------------------------------------------------

\addbibresource{References.bib}

% -- Boxes ---------------------------------------------------------------------

\newtcbox{\codebox}{
  on line,
  arc = 2pt,
  colback = gray!10!white,
  boxsep = 2pt,
  left = 1pt,
  right = 1pt,
  top = 1pt,
  bottom = 1pt,
  boxrule = 0pt,
  bottomrule = 0pt,
  toprule = 0pt
}

\newtcolorbox{code-boxed}{
  on line,
  arc=2pt,
  colback=gray!10!white,
  boxsep=2pt,
  left=1pt,
  right=1pt,
  top=1pt,
  bottom=1pt,
  boxrule=0pt,
  bottomrule=0pt,
  toprule=0pt
}

% -- Code ----------------------------------------------------------------------

\usemintedstyle{rainbow_dash}
\BeforeBeginEnvironment{minted}{\begin{code-boxed}}
\AfterEndEnvironment{minted}{\end{code-boxed}}
\newminted{c}{autogobble}
\newminted{cpp}{autogobble}
\newminted{python}{autogobble}
\newminted{shell}{autogobble}
\newminted{yaml}{autogobble}

% Source: https://tex.stackexchange.com/questions/188003
\makeatletter
\newenvironment{tabminted}{%
  \let\FV@ListVSpace\relax
  \minted
}{%
  \endminted
  \unskip
  \aftergroup\@tabmintedend
}
\newcommand*{\tabminted@finalstrut}[1]{%
  \ifdim\prevdepth>0pt
    \ifdim\dp#1>\prevdepth
      \vskip\dimexpr(\dp#1)-\prevdepth\relax
    \fi
  \else
    \vskip\dimexpr(\dp#1)\relax
  \fi
}
\newcommand*{\@tabmintedend}{%
  \let\@finalstrut\tabminted@finalstrut
}
\makeatother

% -- Captions ------------------------------------------------------------------

\captionsetup{format=hang}

% -- Colors --------------------------------------------------------------------

\definecolor{Aqua}{rgb}{0, 0.56, 1}
\definecolor{Red}{rgb}{0.84, 0.15, 0.19}
\definecolor{Blue}{rgb}{0, 0.41, 0.60}

\definecolor{color02}{rgb}{0.19,0.52,0.58}
\definecolor{color03}{rgb}{0.17,0.36,0.80}
\definecolor{color04}{rgb}{0.35,0.09,0.73}
\definecolor{color05}{rgb}{0.30,0.30,0.30}
\definecolor{color06}{rgb}{0.00,0.50,1.00}
\definecolor{color07}{rgb}{0.00,0.80,0.40}

% -- Floats --------------------------------------------------------------------

\floatplacement{figure}{htbp}
\floatplacement{table}{htbp}

% -- Fonts ---------------------------------------------------------------------

\setmainfont[Mapping=tex-text]{Seravek}
\setsansfont[Mapping=tex-text]{Ubuntu}
\setmonofont[Scale=MatchLowercase]{Menlo}
\setmathfont{texgyrepagella-math.otf}

% -- Hyperref ------------------------------------------------------------------

\hypersetup{
  pdfauthor = {\authorname},
  pdftitle = {\thesistitle},
  pdfsubject = {Comparison of Different Parsing Methods},
  pdfkeywords = {Parsing, Configuration, Elektra},
  colorlinks=true,
  linkcolor=black,
  anchorcolor=black,
  citecolor=Red,
  urlcolor=Red
}

% -- Indentation & Paragraphs --------------------------------------------------

\nonzeroparskip
\setlength{\parindent}{0pt}

% -- Page Numbering ------------------------------------------------------------

\renewcommand{\pagenumbering}[1]{}

% -- Section & Paragraph Style -------------------------------------------------

\setsecnumdepth{subsubsection}
\titleformat{\section}{\large\sffamily\bfseries}{}{0pt}{\thesection~}
  [{\color{Red}\hrule}]
\titleformat{\paragraph}{\bfseries}{}{0pt}{\centering}[\vspace{1pt}]
\titleformat{\subparagraph}{\bfseries}{}{0pt}{}[\vspace{1pt}]

% -- Table of Contents ---------------------------------------------------------

\settocdepth{subsection}

% -- Theorems ------------------------------------------------------------------

\declaretheorem[name={Research Question}]{question}

% -- Macros --------------------------------------------------------------------

\newcommand{\code}[1]{\codebox{\texttt{#1}}}
\newcommand{\cc}[1]{\codebox{\mintinline{c}|#1|}}
\newcommand{\cpp}[1]{\codebox{\mintinline{cpp}|#1|}}
\newcommand{\yaml}[1]{\codebox{\mintinline{yaml}|#1|}}

% -- Glossary ------------------------------------------------------------------

\newacronym{ABNF}{ABNF}{augmented Backus-Naur form}
\newacronym{AFL}{AFL}{American fuzzy lop}
\newacronym{ANTLR}{ANTLR}{Another Tool for Language Recognition}
\newacronym{AST}{AST}{abstract syntax tree}
\newacronym{BNF}{BNF}{Backus-Naur form}
\newacronym{KDB}{KDB}{key database}
\newacronym{NS}{NS}{Namespace}
\newacronym{PEG}{PEG}{Parsing Expression Grammar}
\newacronym{PEGTL}{PEGTL}{Parsing Expression Grammar Template Library}
\newacronym{YAML}{YAML}{YAML Ain't Markup Language}
\newacronym{YAEP}{YAEP}{Yet Another Earley Parser}

\newglossaryentry{collection}{
    name = collection,
    description = {A collection is a data structure used to store multiple items. In YAML the term collection is used to refer to both sequences and mappings}
}

\newglossaryentry{simpleKey}{
    name = simple key,
    description = {A simple YAML key is a key that does not start with the explicit mapping start symbol “\yaml{?}”}
}

\newcommand{\GlsShort}[1]
  {\setacronymstyle{short-long}\gls{#1}\setacronymstyle{long-short}}

\makeindex
\makeglossaries

% -- Document ------------------------------------------------------------------

\begin{document}

\frontmatter

\addtitlepage{naustrian}
\addtitlepage{english}
\addstatementpage

\selectlanguage{english}

\tableofcontents

\mainmatter

\chapter{Introduction}

\section{Motivation \& Problem Statement}

\emph{Parsing} is the process of structuring linear input according to a grammar~\cite{grune2008parsing}. Often the output of this structuring process is a data structure~\cite{wikipedia2019Parser}. Research around this topic focuses on how to process the input, how to handle erroneous input and the general computational complexity of algorithms to handle certain kind of \emph{formal languages}.

\begin{figure}[H]
  \centering
    \includegraphics[width=.8\textwidth]{Parsing}
  \caption{Simplified view of the parsing process}
\end{figure}

This thesis concerns itself with the parsing process of languages that are able to express configuration data (e.g. INI, TOML, \glstext{YAML}). These languages form an interesting subpart of formal languages, since accessing key-value based persistent configuration data is a common task for many computer programs.

Just like people disagree about the “best” configuration format, there is currently no consensus, as to which is the ideal way to parse configuration data. There are many possible ways to parse and store data. Notable examples include:

\begin{itemize}
  \item bidirectional programming~\cite{foster2005combinators, bohannon2006relational, lutterkort2008augeas, ko2016bigul, raab2016improving},
  \item code produced by a parser generator~\cite{denny2008ielr, parr2014adaptive, warth2016modular, bates2017aprt},
  \item Serialization libraries~\cite{sumaray2012cds, pacini2015performance}, and
  \item Hand-written parsers~\cite{myers2008cparser, bendersky2012clang}
\end{itemize}

. Currently the possibilities to compare different parsing techniques are limited. The naive approach would be to just run different parsers on the same data. In practice however, this approach is not usable, since parser tools tend to produce very different data structures. Some of them do not produce data structures at all, instead they let the user specify subroutines that should be called when the parser matches parts of the grammar.

As part of this thesis we will tackle this problem, using different parsing techniques within a common configuration framework. This integration eliminates the problem of comparing the parsing process under different circumstances, since the data structures the parsers create will always be the same. We will use \href{http://web.libelektra.org}{Elektra}, a key-value database, as configuration framework. Elektra’s storage plugin interface will act as foundation for the parsing process. In the end the thesis should provide answers about which parsing techniques provide an ideal balance between performance and usability.

\section{Aim of the Work}
\label{sec:aim_of_the_work}

Elektra~\cite{raab2010modular, raab2017context} is a plugin based framework that stores configuration parameters in a \gls{KDB}. Elektra reads and stores configuration data via so-called \emph{storage plugins}.

\begin{figure}[H]
  \centering
    \includegraphics[width=.3\textwidth]{Elektra}
  \caption{Architecture diagram of Elektra}
\end{figure}

As part of this thesis we compare various ways of parsing. For that purpose we wrote and generated parsing code for different storage plugins. All of these storage plugins parse a minimal subset of \glstext{YAML}, a human readable configuration language. We looked at the following parsing technologies:

\begin{itemize}
  \item handwritten parser (recursive descent),
  \item \glstext{ALL(*)} parser generator (\href{http://www.antlr.org}{ANTLR}),
  \item LR parser generator (\href{https://www.gnu.org/software/bison}{Bison}),
  \item Earley parser (\href{https://github.com/vnmakarov/yaep}{YAEP}),
  \item \glstext{PEG} parser (\href{https://github.com/taocpp/PEGTL}{PEGTL}),
  \item parser combinator (\href{https://github.com/orangeduck/mpc}{mpc}), and
  \item bidirectional programming (\href{http://augeas.net}{Augeas})
\end{itemize}

. We compare the parsing code according to the following criteria:

\begin{itemize}
  \item runtime performance,
  \item memory usage,
  \item code size,
  \item overall code complexity,
  \item ease of extensibility and composability, and
  \item error reporting
\end{itemize}

. In the scope of the above comparison we answer the questions below.

\begin{restatable}{question}{speed}
  \label{que:speed}
  How does the theoretic runtime complexity of the parsing methods compare to the actual measured runtime of the parsing code?
\end{restatable}

\begin{restatable}{question}{closeness}
  \label{que:closeness}
  Which parsing technique allows us to stay closest to the definition of the configuration language? Does staying close to the given definition,
  allow us to extend and improve the parser and its support code more easily?
\end{restatable}

\section{Methodological Approach}

The methodological approach for this thesis consists of the steps below.

\begin{description}[style=multiline, leftmargin=3.2cm, font=\bfseries]

  \item[Literature Review] We determined the current status of parsing techniques suitable for configuration file parsing. We then chose appropriate libraries for the parsing techniques listed in the section “\nameref{sec:aim_of_the_work}”.

  \item[Discussion] To determine a minimal usable subset of \glstext{YAML} we discussed common features required for a new Elektra storage plugin with some of the current developers as part of an presentation and subsequent discussion.

  \item[Implementation] We wrote parsing code that handles our minimal \glstext{YAML} subset. In this phase we also added other necessary support code to Elektra.

  \item[Comparison] As noted in “\nameref{sec:aim_of_the_work}” we evaluated the different implementations of our minimal \glstext{YAML} subset parsers.

  \begin{description}
    \item[Runtime Benchmark:] For the runtime comparison we use benchmarks to determine the speed of the different parsing code. In this part of the thesis, we answer \Cref{que:speed}.

    \item[Memory Profiling:] For the memory comparison we use a memory profiling tool to determine the heap memory usage of the \glstext{YAML} plugins.

    \item[Code Count:] We counted the number of code lines with a code line counting tool. This method allow us to consider only actual code, ignoring blank lines and comments.

    \item[Complexity Measurement:] We measured the \gls{CC} of the code using a static analyzer.

    \item[Extensibility \& Composability Check:] To analyze the extensibility and composability of the parsing code we looked at the code difference of commits for certain features and bug fixes. We counted the amount of updated code lines to determine the extension effort. This measurement, together with a comparison between the grammar specification of \glstext{YAML} and the code created in this thesis, will help us to determine the answer for \Cref{que:closeness}.

    \item[Error Reporting:] To determine the quality of the error messages we created erroneous files and compared the quality of the resulting error output.
  \end{description}

\end{description}

\chapter{Background}

In the first part of the thesis we looked at the current state of parsing techniques according to the literature.

\section{State of the Art}
\label{sec:state_of_the_art}

The book \citetitle{grune2007parsing}~\cite{grune2007parsing} provides a good overview of various up-to-date parsing algorithms. It covers the most popular techniques (such as LL- and LR-Parsing~\cite{knuth1965translation}) and also less well known methods up to 2007. The book also describes various classification possibilities for parsing techniques \cite[p. 85]{grune2007parsing}. The most common classification is the division into bottom-up and top-down parsers.

\begin{figure}[H]
   \centering
   \subfloat[A top-down parser predicts and matches rules from the start symbol downwards.]{\includegraphics[width=.3\textwidth]{Top-Down}}
   \qquad
   \subfloat[A bottom-up parser recognizes text starting with the terminal symbols.]{\includegraphics[width=.3\textwidth]{Bottom-Up}}
   \caption{Matching in top-down and bottom-up parsers}
 \end{figure}

In \emph{top-down parsing} the parser starts with a hypothesis about the structure of the given data. The parser then tries to predict and match parts of the structure, starting from larger parts working its way down-to smaller elements.

We can further categorize parsing into \emph{directional} and \emph{non-directional} methods. Directional methods read the input from left to right, while non-directional methods can use an arbitrary order. This implies that directional methods are simpler and faster, but less powerful, than their non-directional counterparts. As part of this thesis we only consider directional methods, since they are faster and powerful enough to parse configuration data.

One of the most popular directional top-down methods is \emph{LL parsing}. While this technique is quite old – \citetitle{grune2007parsing} (p. 584) mentions a paper from 1961 belonging to the LL category – it is still actively used and researched. The basic idea behind LL parsing is simple: Begin with the start symbol of the grammar and the first character of the text. Then predict the next grammar rule, looking at the text to the right of the current position. We can categorize the technique further depending on the number of characters/tokens the parser uses to predict the next rule. If the parser uses one token of look-ahead we speak of an LL(1) parser, if it uses k tokens of lookahead we speak of an LL(k) parser~\cite{rosenkrantz1969properties}.

Two common methods to create an LL parser are:

\begin{enumerate}

  \item Implement the parser code using a set of mutually recursive procedures (Recursive Descent Parser). The code for this is either written by hand or produced by a parser generator such as \GlsShort{ANTLR}~\cite{parr2013recursive}.

  \item Use a parser generator to create a table-based parser.

\end{enumerate}

Examples of popular active projects that use a handwritten recursive descent parser include \href{http://clang.llvm.org}{clang}~\cite{bendersky2012clang} and \href{http://gcc.gnu.org}{GCC}~\cite{myers2008cparser}. The \href{http://www.antlr.org/about.html}{about page} for \gls{ANTLR} mentions some projects that use its generated recursive descent parsers. The list includes Twitter, wich uses ANTLR for query parsing and parsers for the languages used in the Apache Hadoop projects Hive and Pig~\cite{parr2013definitive}.

Some of the latest research developments in LL-parsing include LL(*) parsing~\cite{parr2011ll} and its successor Adaptive LL(*)~\cite{parr2014adaptive} (ALL(*)). Both of these algorithms use dynamic lookahead~\cite[p. 1]{parr2011ll}. While LL(*) parsing uses a static algorithm for rule prediction, ALL(*) analyses the input at run-time to improve prediction. As consequence of this parsers using the ALL(*) algorithm will be faster after an initial warm-up phase~\cite[p. 3]{parr2014adaptive}. LL(*) is part of \gls{ANTLR}~3~\cite[p. 3]{parr2014adaptive}, while \gls{ANTLR}~4 uses Adaptive LL parsing.

As we already mentioned before, the other popular parsing technique besides top-down-parsing is \emph{bottom-up parsing}. In bottom-up parsing the parser builds a structure starting with the smallest elements of the grammar (terminals). The parser then combines these elements into larger parts. One of the earliest entries in the \emph{linear bottom-up} parser category is the LR(k) parser~\cite{knuth1965translation}. Just like in LL(k) parsing, k specifies the number of lookahead symbols the parser uses.

\begin{sloppypar}
Unlike LL parsers, LR parsers are usually not created by hand, but generated by a parsing tool such as \href{https://www.gnu.org/software/bison}{bison} or \href{http://dinosaur.compilertools.net/yacc}{yacc}. Since LR(k) tables are very large, even for a small numbers of k, the parser tools mentioned before generate less powerful but smaller and faster LALR(k)~\cite{deremer1969practical} parsers.
\end{sloppypar}

LR(k) parsers are able to handle more grammars, than LL(k) parsers for the same constant k~\cite[section “Lookahead”]{haberman2013ll}. However, LR parsers are still not able to use ambiguous grammars. For this purpose \citeauthor{lang1974deterministic} describes the Generalized LR (GLR)~\cite{lang1974deterministic} method that is also able to handle these types of grammars. GLR parser are sometimes also called Tomita parsers~\cite{tomita1985efficient} after the author that described the first implementation of a generalized LR parser.

Recent research in the space of directional bottom-up parsing includes improved versions of techniques that are almost as powerful as canonical LR(1). One of the most promising methods is IELR(1)~\cite{denny2008ielr}. The advantage of IELR(1) over LALR(1) is that it handles conflict resolution better. Parser tools such as bison use conflict resolution to handle non-LR grammars, i.e. grammars that contain rules where the parser is not able to decide what to do next. To handle these types of conflicts the grammar designer manually specifies which decision the parser should take. The current version of the parsing tool bison supports an experimental version of IELR(1).

Most parsing techniques can be categorized as either top-down or bottom-up. However, some techniques use a combination of both approaches. Others are usually not listed under one of the label top-down or bottom up, because they provide other features that the designer of these parsers deem more important, or they use features that do not fit well within either of these groups. In the remainder of this section we will discuss some of these techniques.

A method that can be categorized as either top-down technique with bottom-up recognition, or bottom-up technique with a top-down component~\cite[p. 206]{grune2007parsing} is \emph{Earley Parsing}~\cite{earley1970efficient}. Earley parsing is able to handle any context free grammar. This means the technique is as powerful as GLR parsing. This advantage comes at the cost of run-time. While LL parsing and LR parsing run in linear time depending on the length of the input ($O(n)$), Earley Parsing has an upper boundary of $O(n³)$. However, in \citeyear{leo1991general} \citeauthor{leo1991general} showed that an improved version of the algorithm handles most LR(k) grammars in linear time~\cite{kegler2011marpa, leo1991general, wikipedia2016Earley}. In \citeyear{aycock2002practical} \citeauthor{aycock2002practical} described improvements to the algorithm. Their version of Earley Parsing boosts the run-time in cases where the grammar contains nullable (empty) grammar rules. Recently \citeauthor{kegler2011marpa} incorporated the changes proposed by \citeauthor{leo1991general}, \citeauthor{aycock2002practical} in \href{http://savage.net.au/Marpa.html}{Marpa}~\cite{kegler2011marpa}.

\begin{figure}
  \centering
    \includegraphics[width=0.45\textwidth]{Chomsky-PEGs}
  \caption{Both the Chomsky grammar on the left and the PEG on the right describe the same language $ \{ aⁿ bⁿ | n ≥ 0 \} $}
\end{figure}

All the methods we mentioned until now work with a description that is based on a (context-free) Chomsky grammar. These grammars describe a way to \emph{generate} words and sentences of a given language. Another way to specify the structure of a language is to give a description on how to \emph{recognize}~\cite[p. 506]{grune2007parsing} the words and sentences of a language. One popular recognition system are \glspl{PEG}. They were introduced by \citeauthor{ford2004parsing} in the paper \citetitle{ford2004parsing}~\cite{ford2004parsing}. \citeauthor{ford2002packrat} also describes how to write an efficient (top-down) parser that handles these types of grammars in linear time~\cite{ford2002packrat}. This method, called Packrat parsing, uses a specialized version of memoziation~\cite[p. 1]{ford2002packrat} to save intermediate results of the parsing process.

Just like Packrat parsing, \emph{combinatory parsing}~\cite{frost1992constructing, hutton1992higher} specifies a method to create recursive descent parsers. As the name suggests, the focus in combinatory parsing is the composability of parsers. The technique is usually used in functional programming languages, such as Haskell. These languages support higher order functions, i.e. functions that take other functions as their parameters~\cite[p. 564]{grune2007parsing}. In combinatory parsing the parser creator typically starts by specifying parser (functions) for the simplest parts of the grammar (terminals). She or he then goes on to combine these simpler parsers into more powerful parsers for more complex rules (non-terminals). Combinatory parsing has similar problems as other top-down techniques such as LL parsing. One of these problems are left recursive grammar rules, i.e. rules that include references to themselves in the leftmost part of the right-hand side. Recently \citeauthor{frost2007modular} described a method to handle left recursive rules in combinatory parsing efficiently in the article \citetitle{frost2007modular}~\cite{frost2007modular}.

A method that is not a parsing technique per se, but a way to specify conversions of data from a source structure to a target structure and back is \emph{bidirectional programming}~\cite{foster2005combinators, bohannon2006relational}. The specification that allows this conversion is called a lens~\cite{foster2005combinators}.

\begin{figure}[H]
  \centering
    \includegraphics[width=.4\textwidth]{Lens.pdf}
  \caption{Lenses provide a way to both parse (get) and write (put) structured data \newline (Source: \href{http://www.seas.upenn.edu/~harmony/manual.pdf}{Boomerang Programmer’s Manual})}
\end{figure}

A programming language used to specify such lenses is Boomerang~\cite{bohannon2008boomerang}. The research of the Boomerang project lead to the creation of another project that uses lenses to parse configuration data: \href{http://augeas.net}{Augeas}. Augeas converts configuration data into a tree like representation. \citeauthor{berlakovich2016universal} implemented an Augeas plugin for Elektra as part of his Bachelor thesis~\cite{berlakovich2016universal}.

\chapter{Approach}

The \glstext{YAML} standard is extensive~\cite{ben2009yaml}. The document describing the serialization language includes about 200 parameterized \gls{BNF} grammar rules. To simplify the parser development we decided to first determine a useful subset of \glstext{YAML}. For this purpose we discussed the language with other Elektra developers. The first part of this chapter contains a review detailing this discussion.

In the second part we describe how we mapped Elektra’s two basic types, \cc{Key} and \cc{KeySet} to \glstext{YAML}.

\section{Discussion}

In the discussion 9 participants answered questions about the usefulness of certain \glstext{YAML} features for \href{https://www.libelektra.org}{Elektra}. We introduced the \glstext{YAML} syntax in a \href{https://github.com/sanssecours/YAML-Presentation/releases/download/v1.0/Presentation.pdf}{presentation}. After we talked about a certain part of YAML, we answered questions the participants had about the information presented so far. Afterwards we asked the participants to fill in parts of a \href{https://github.com/sanssecours/YAML-Presentation/blob/master/Questionnaire.md}{questionnaire} about the newly introduced feature set. The questionnaire consisted of checkboxes for each feature. A checked box indicates that the participant considers a feature useful for Elektra.

\subsection{Participants}

All of the participants were at least partially familiar with Elektra. Some also had previous experience with \glstext{YAML}. Seven of them listened to the presentation, while one participant was late and another one participated via email. The email participant received a copy of the presentation slides and the questionnaire.

\subsection{Results}

In the following bar charts the term “Yes” refers to a checked box for the specific feature. The term “?” means that the participant did not know enough about a part of \glstext{YAML} and therefore marked the checkbox for one feature, or the heading for multiple features, with a question mark. The value before the term “No” specifies the number of unchecked boxes minus the number of boxes marked with “?”.

\subsubsection{Scalars}

\paragraph{Flow Scalars}

\begin{figure}[H]
  \begin{minipage}[t]{0.48\textwidth}
    \vspace{0pt}
    \begin{bchart}[max=9, width=0.85\textwidth]
      \bcbar[text=3, value=Yes, color=orange]{3}
      \bcbar[text=6, value=No, color=Aqua]{6}
    \end{bchart}
  \end{minipage}
  \begin{minipage}[t]{0.48\textwidth}
    \vspace{0pt}
    \begin{yamlcode}
      Plain String
    \end{yamlcode}
  \end{minipage}
  \caption{Plain Flow Scalar}
\end{figure}

\begin{figure}[H]
  \begin{minipage}[t]{0.48\textwidth}
    \vspace{0pt}
    \begin{bchart}[max=9, width=0.85\textwidth]
      \bcbar[text=2, value=Yes, color=orange]{2}
      \bcbar[text=7, value=No, color=Aqua]{7}
    \end{bchart}
  \end{minipage}
  \begin{minipage}[t]{0.48\textwidth}
    \vspace{0pt}
    \begin{yamlcode}
      'Single Quoted ''String'''
    \end{yamlcode}
  \end{minipage}
  \caption{Single Quoted Flow Scalar}
\end{figure}

\begin{figure}[H]
  \begin{minipage}[t]{0.48\textwidth}
    \vspace{0pt}
    \begin{bchart}[max=9, width=0.85\textwidth]
      \bcbar[text=8, value=Yes, color=orange]{8}
      \bcbar[text=1, value=No, color=Aqua]{1}
    \end{bchart}
  \end{minipage}
  \begin{minipage}[t]{0.48\textwidth}
    \vspace{0pt}
    \begin{yamlcode}
      "Double\n Quoted\n \"String\""
    \end{yamlcode}
  \end{minipage}
  \caption{Double Quoted Flow Scalar}
\end{figure}

\paragraph{Block Scalars}

\begin{figure}[H]
  \begin{minipage}[t]{0.48\textwidth}
    \vspace{0pt}
    \begin{bchart}[max=9, width=0.85\textwidth]
      \bcbar[text=2, value=Yes, color=orange]{2}
      \bcbar[text=6, value=No, color=Aqua]{6}
      \bcbar[text=1, value=?, color=DarkTurquoise]{1}
    \end{bchart}
  \end{minipage}
  \begin{minipage}[t]{0.48\textwidth}
    \vspace{0pt}
    \begin{yamlcode}
      > # "Folded Style"
        Folded
        Style
    \end{yamlcode}
  \end{minipage}
  \caption{Folded Block Scalar}
\end{figure}

\begin{figure}[H]
  \begin{minipage}[t]{0.48\textwidth}
    \vspace{0pt}
    \begin{bchart}[max=9, width=0.85\textwidth]
      \bcbar[text=2, value=Yes, color=orange]{2}
      \bcbar[text=6, value=No, color=Aqua]{6}
      \bcbar[text=1, value=?, color=DarkTurquoise]{1}
    \end{bchart}
  \end{minipage}
  \begin{minipage}[t]{0.48\textwidth}
    \vspace{0pt}
    \begin{yamlcode}
      | # "Literal\nStyle"
        Literal
        Style
    \end{yamlcode}
  \end{minipage}
  \caption{Literal Block Scalar}
\end{figure}

\begin{figure}[H]
  \begin{minipage}[t]{0.48\textwidth}
    \vspace{0pt}
    \begin{bchart}[max=9, width=0.85\textwidth]
      \bcbar[text=1, value=Yes, color=orange]{1}
      \bcbar[text=7, value=No, color=Aqua]{7}
      \bcbar[text=1, value=?, color=DarkTurquoise]{1}
    \end{bchart}
  \end{minipage}
  \begin{minipage}[t]{0.48\textwidth}
    \vspace{0pt}
    \begin{yamlcode*}{showspaces, spacecolor = lightgray, space=·}
      >1 # "  1 Space Indentation"
         1 Space Indentation
    \end{yamlcode*}
  \end{minipage}
  \caption{Indentation Header}
\end{figure}

\begin{figure}[H]
  \begin{minipage}[t]{0.48\textwidth}
    \vspace{0pt}
    \begin{bchart}[max=9, width=0.85\textwidth]
      \bcbar[text=0, value=$\quad$Yes, color=orange]{0}
      \bcbar[text=8, value=No, color=Aqua]{8}
      \bcbar[text=1, value=?, color=DarkTurquoise]{1}
    \end{bchart}
  \end{minipage}
  \begin{minipage}[t]{0.48\textwidth}
    \vspace{0pt}
    \begin{yamlcode*}{showspaces, spacecolor = lightgray, space=·, escapeinside=||}
      >- # "No Trailing Whitespace"
         No Trailing Whitespace
        | |
        | |
      # ↑ Newlines Above Stripped
    \end{yamlcode*}
  \end{minipage}
  \caption{Chomping Header}
\end{figure}

\subsubsection{Lists}

\begin{figure}[H]
  \begin{minipage}[t]{0.48\textwidth}
    \vspace{0pt}
    \begin{bchart}[max=9, width=0.85\textwidth]
      \bcbar[text=5, value=Yes, color=orange]{5}
      \bcbar[text=4, value=No, color=Aqua]{4}
    \end{bchart}
  \end{minipage}
  \begin{minipage}[t]{0.48\textwidth}
    \vspace{0pt}
    \begin{yamlcode}
      [🍎, 🍊,
        [Sugar, Eggs, Chocolate]
      ]
    \end{yamlcode}
  \end{minipage}
  \caption{Flow Style}
\end{figure}

\begin{figure}[H]
  \begin{minipage}[t]{0.48\textwidth}
    \vspace{0pt}
    \begin{bchart}[max=9, width=0.85\textwidth]
      \bcbar[text=7, value=Yes, color=orange]{7}
      \bcbar[text=2, value=No, color=Aqua]{2}
    \end{bchart}
  \end{minipage}
  \begin{minipage}[t]{0.48\textwidth}
    \vspace{0pt}
    \begin{yamlcode}
      - 🍎
      - 🍊
      - - Sugar
        - Eggs
        - Chocolate
    \end{yamlcode}
  \end{minipage}
  \caption{Block Style}
\end{figure}

\subsubsection{Mappings}

\begin{figure}[H]
  \begin{minipage}[t]{0.48\textwidth}
    \vspace{0pt}
    \begin{bchart}[max=9, width=0.85\textwidth]
      \bcbar[text=5, value=Yes, color=orange]{5}
      \bcbar[text=4, value=No, color=Aqua]{4}
    \end{bchart}
  \end{minipage}
  \begin{minipage}[t]{0.48\textwidth}
    \vspace{0pt}
    \begin{yamlcode}
      { Austria: Vienna,
        South Africa: {
          Executive: Pretoria,
          Judicial: Bloemfontein,
          Legislative: Cape Town }
      }
    \end{yamlcode}
  \end{minipage}
  \caption{Flow Style}
\end{figure}

\begin{figure}[H]
  \begin{minipage}[t]{0.48\textwidth}
    \vspace{0pt}
    \begin{bchart}[max=9, width=0.85\textwidth]
      \bcbar[text=7, value=Yes, color=orange]{7}
      \bcbar[text=2, value=No, color=Aqua]{2}
    \end{bchart}
  \end{minipage}
  \begin{minipage}[t]{0.48\textwidth}
    \vspace{0pt}
    \begin{yamlcode}
      Austria: Vienna
      South Africa:
        Executive:   Pretoria
        Judicial:    Bloemfontein
        Legislative: Cape Town
    \end{yamlcode}
  \end{minipage}
  \caption{Block Style}
\end{figure}

\begin{figure}[H]
  \begin{minipage}[t]{0.48\textwidth}
    \vspace{0pt}
    \begin{bchart}[max=9, width=0.85\textwidth]
      \bcbar[text=0, value=$\quad$Yes, color=orange]{0}
      \bcbar[text=9, value=No, color=Aqua]{9}
    \end{bchart}
  \end{minipage}
  \begin{minipage}[t]{0.48\textwidth}
    \vspace{0pt}
    \begin{yamlcode}
      ?
      - { 'pretty': complex key }
      - - 😱
      - Still part of the key
      : value
    \end{yamlcode}
  \end{minipage}
  \caption{Support for Complex Keys}
\end{figure}

\subsubsection{Multiple Documents}

\begin{figure}[H]
  \begin{minipage}[t]{0.48\textwidth}
    \vspace{0pt}
    \begin{bchart}[max=9, width=0.85\textwidth]
      \bcbar[text=0, value=$\quad$Yes, color=orange]{0}
      \bcbar[text=9, value=No, color=Aqua]{9}
    \end{bchart}
  \end{minipage}
  \begin{minipage}[t]{0.48\textwidth}
    \vspace{0pt}
    \begin{yamlcode}
      "Hello First Document"
      ...
      'Second Document'
      ...
      Third Document
    \end{yamlcode}
  \end{minipage}
  \caption{Support Streams}
\end{figure}

\subsubsection{Types}

\paragraph{Directives}

\begin{figure}[H]
  \begin{minipage}[t]{0.48\textwidth}
    \vspace{0pt}
    \begin{bchart}[max=9, width=0.85\textwidth]
      \bcbar[text=1, value=Yes, color=orange]{1}
      \bcbar[text=7, value=No, color=Aqua]{7}
      \bcbar[text=1, value=?, color=DarkTurquoise]{1}
    \end{bchart}
  \end{minipage}
  \begin{minipage}[t]{0.48\textwidth}
    \vspace{0pt}
    \begin{yamlcode}
      %YAML 1.2
    \end{yamlcode}
  \end{minipage}
  \caption{\glstext{YAML} Version}
\end{figure}

\begin{figure}[H]
  \begin{minipage}[t]{0.48\textwidth}
    \vspace{0pt}
    \begin{bchart}[max=9, width=0.85\textwidth]
      \bcbar[text=3, value=Yes, color=orange]{3}
      \bcbar[text=5, value=No, color=Aqua]{5}
      \bcbar[text=1, value=?, color=DarkTurquoise]{1}
    \end{bchart}
  \end{minipage}
  \begin{minipage}[t]{0.48\textwidth}
    \vspace{0pt}
    \begin{yamlcode}
      %TAG !      tag:yaml.org,2002:
      %TAG !!     tag:yaml.org,2002:
      %TAG !name! tag:yaml.org,2002:
      ---
    \end{yamlcode}
  \end{minipage}
  \caption{Tag Handle Definition}
\end{figure}

\begin{figure}[H]
  \begin{minipage}[t]{0.48\textwidth}
    \vspace{0pt}
    \begin{bchart}[max=9, width=0.85\textwidth]
      \bcbar[text=2, value=Yes, color=orange]{2}
      \bcbar[text=6, value=No, color=Aqua]{6}
      \bcbar[text=1, value=?, color=DarkTurquoise]{1}
    \end{bchart}
  \end{minipage}
  \begin{minipage}[t]{0.48\textwidth}
    \vspace{0pt}
    \begin{yamlcode}
      %TAG !name! tag:yaml.org,2002:
      ---
      !name!str 6 # "6"
    \end{yamlcode}
  \end{minipage}
  \caption{Named Tag Handle}
\end{figure}

\paragraph{Tags}

\subparagraph{Tag Shorthands}

\begin{figure}[H]
  \begin{minipage}[t]{0.48\textwidth}
    \vspace{0pt}
    \begin{bchart}[max=9, width=0.85\textwidth]
      \bcbar[text=4, value=Yes, color=orange]{4}
      \bcbar[text=4, value=No, color=Aqua]{4}
      \bcbar[text=1, value=?, color=DarkTurquoise]{1}
      \bcxlabel{}
    \end{bchart}
  \end{minipage}
  \begin{minipage}[t]{0.48\textwidth}
    \vspace{0pt}
    \begin{yamlcode}
      !suffix value
    \end{yamlcode}
  \end{minipage}
  \caption{Primary Tag Handle}
\end{figure}

\begin{figure}[H]
  \begin{minipage}[t]{0.48\textwidth}
    \vspace{0pt}
    \begin{bchart}[max=9, width=0.85\textwidth]
      \bcbar[text=3, value=Yes, color=orange]{3}
      \bcbar[text=5, value=No, color=Aqua]{5}
      \bcbar[text=1, value=?, color=DarkTurquoise]{1}
    \end{bchart}
  \end{minipage}
  \begin{minipage}[t]{0.48\textwidth}
    \vspace{0pt}
    \begin{yamlcode}
      !!suffix value
    \end{yamlcode}
  \end{minipage}
  \caption{Secondary Tag Handle}
\end{figure}

\begin{figure}[H]
\subparagraph{Verbatim Tags}
  \begin{minipage}[t]{0.48\textwidth}
    \vspace{0pt}
    \begin{bchart}[max=9, width=0.85\textwidth]
      \bcbar[text=0, value=$\quad$Yes, color=orange]{0}
      \bcbar[text=8, value=No, color=Aqua]{8}
      \bcbar[text=1, value=?, color=DarkTurquoise]{1}
    \end{bchart}
  \end{minipage}
  \begin{minipage}[t]{0.48\textwidth}
    \vspace{0pt}
    \begin{yamlcode}
      !<!ruby/object:Set> value
    \end{yamlcode}
  \end{minipage}
  \caption{Local Verbatim Tags}
\end{figure}

\begin{figure}[H]
  \begin{minipage}[t]{0.48\textwidth}
    \vspace{0pt}
    \begin{bchart}[max=9, width=0.85\textwidth]
      \bcbar[text=0, value=$\quad$Yes, color=orange]{0}
      \bcbar[text=8, value=No, color=Aqua]{8}
      \bcbar[text=1, value=?, color=DarkTurquoise]{1}
    \end{bchart}
  \end{minipage}
  \begin{minipage}[t]{0.48\textwidth}
    \vspace{0pt}
    \begin{yamlcode}
      !<tag:yaml.org,2002:str> value
    \end{yamlcode}
  \end{minipage}
  \caption{Global Verbatim Tags}
\end{figure}

\subparagraph{Other Tags}

\begin{figure}[H]
  \begin{minipage}[t]{0.48\textwidth}
    \vspace{0pt}
    \begin{bchart}[max=9, width=0.85\textwidth]
      \bcbar[text=0, value=$\quad$Yes, color=orange]{0}
      \bcbar[text=8, value=No, color=Aqua]{8}
      \bcbar[text=1, value=?, color=DarkTurquoise]{1}
    \end{bchart}
  \end{minipage}
  \begin{minipage}[t]{0.48\textwidth}
    \vspace{0pt}
    \begin{yamlcode}
      ! value
    \end{yamlcode}
  \end{minipage}
  \caption{Non-Specific Tag}
\end{figure}

\paragraph{Schemas}

\textbf{Remark:} One participant checked the box for the core schema without ticking the boxes for the failsafe and JSON schema. Since the core schema is an extended superset of the other two schemas, we counted the participants answers as a “Yes” vote for the failsafe and JSON schema.

\begin{figure}[H]
  \begin{minipage}[t]{0.48\textwidth}
    \vspace{0pt}
    \begin{bchart}[max=9, width=0.85\textwidth]
      \bcbar[text=5, value=Yes, color=orange]{5}
      \bcbar[text=3, value=No, color=Aqua]{3}
      \bcbar[text=1, value=?, color=DarkTurquoise]{1}
      \bcxlabel{}
    \end{bchart}
  \end{minipage}
  \begin{minipage}[t]{0.48\textwidth}
    \vspace{0pt}
    \begin{itemize}
      \item String
      \item Sequence
      \item Map
    \end{itemize}
  \end{minipage}
  \caption{Failsafe Schema}
\end{figure}

\begin{figure}[H]
  \begin{minipage}[t]{0.48\textwidth}
    \vspace{0pt}
    \begin{bchart}[max=9, width=0.85\textwidth]
      \bcbar[text=5, value=Yes, color=orange]{5}
      \bcbar[text=3, value=No, color=Aqua]{3}
      \bcbar[text=1, value=?, color=DarkTurquoise]{1}
    \end{bchart}
  \end{minipage}
  \begin{minipage}[t]{0.48\textwidth}
    \vspace{0pt}
    Failsafe Schema + JSON Types:
    \begin{minipage}[t]{2cm}
      \begin{itemize}[leftmargin=*]
        \item Null
        \item Boolean
        \item Integer
        \item Float
      \end{itemize}
    \end{minipage}
  \end{minipage}
  \caption{JSON Schema}
\end{figure}

\begin{figure}[H]
  \begin{minipage}[t]{0.48\textwidth}
    \vspace{0pt}
    \begin{bchart}[max=9, width=0.85\textwidth]
      \bcbar[text=3, value=Yes, color=orange]{3}
      \bcbar[text=5, value=No, color=Aqua]{5}
      \bcbar[text=1, value=?, color=DarkTurquoise]{1}
    \end{bchart}
  \end{minipage}
  \begin{minipage}[t]{0.48\textwidth}
    \vspace{0pt}
    JSON Schema and
      \vspace{-0.5cm}
      \begin{itemize}
        \item Octal/Hex: \yaml{0o123}, \yaml{0xfefe}
        \item Multiple Notations for same value:
              \yaml{null}, \yaml{Null}, \yaml{~}
      \end{itemize}
  \end{minipage}
  \caption{Core Schema}
\end{figure}

\begin{figure}[H]
  \begin{minipage}[t]{0.48\textwidth}
    \vspace{0pt}
    \begin{bchart}[max=9, width=0.85\textwidth]
      \bcbar[text=3, value=Yes, color=orange]{3}
      \bcbar[text=5, value=No, color=Aqua]{5}
      \bcbar[text=1, value=?, color=DarkTurquoise]{1}
    \end{bchart}
  \end{minipage}
  \begin{minipage}[t]{0.48\textwidth}
    \vspace{0pt}
    \begin{itemize}
      \item Ordered Map
      \item Set
      \item Binary
      \item Time
      \item …
    \end{itemize}
  \end{minipage}
  \caption{Additional Types}
\end{figure}

\subparagraph{Which Additional Types:}
\begin{itemize}
  \item “” (No answer)
  \item “binary”
  \item “date (but implemented in plugins)”
\end{itemize}

\subsubsection{References}

\begin{figure}[H]
  \begin{minipage}[t]{0.48\textwidth}
    \vspace{0pt}
    \begin{bchart}[max=9, width=0.85\textwidth]
      \bcbar[text=7, value=Yes, color=orange]{7}
      \bcbar[text=2, value=No, color=Aqua]{2}
    \end{bchart}
  \end{minipage}
  \begin{minipage}[t]{0.48\textwidth}
    \vspace{0pt}
    \begin{yamlcode}
      flowers: &flowers
        🌳🌸🌼
      garden:
        - *flowers # 🌳🌸🌼
        - *flowers # 🌳🌸🌼
    \end{yamlcode}
  \end{minipage}
  \caption{Support Anchors \& Aliases}
\end{figure}

\subsection{Interpretation}

The results of the survey showed that the participants preferred double quoted flow scalars over single quoted and plain scalars. A reasons for this could be that those scalars are familiar from other languages such as C, and that they are able to express arbitrary data. Asked about block scalar styles most of the Elektra developers did not think that any of the two styles were necessary.

In contrast to the decision about block scalars the participants preferred the block styles of sequences and mappings (collections) over the respective flow style. However, they also decided that a minimally useful \glstext{YAML} subset should include flow collections.

The Elektra developers decided against most of the specialized type features of \glstext{YAML}. Only the result count for and against primary tag handles resulted in a draw.

The questions about general type support (schemas) showed that a minimal \glstext{YAML} subset should include all types of the JSON Schema.

One of the few specialized features deemed necessary by the participants were anchors and aliases. These two elements can be used to reference the same data multiple times in the same document.

\subsubsection{Summary/Decision}
\label{sec:discussion_summary_decision}

The list below contains a summary of the \glstext{YAML} features that should be part of a minimal \glstext{YAML} subset according to the results of the discussion:

\begin{itemize}
  \item Double quoted flow scalars
  \item Block and flow collections
  \item JSON schema
  \item Primary tag handle
  \item References
\end{itemize}

. In the end we decided to drop some features in the list above and implement the following list of items for the \glstext{YAML} subset:

\begin{itemize}
  \item Double quoted scalars
  \item Single quoted scalars
  \item Plain flow scalars
  \item Block collections
  \item Core schema (no tag support)
\end{itemize}

. The most notable difference to the result of the discussion are the missing support for data types, references and flow collections.

\textbf{Typing} is an interesting feature. From the perspective of the parser, adding support for tags should be easy. However, the translation of \glstext{YAML}’s data types to the ones supported by Elektra, is not trivial, and hence we decided to not support tags.

We also decided against supporting \textbf{references}, since this feature allow us to specify cyclic data structures, something not supported by Elektra’s \nameref{sec:keyset} data structure.

\textbf{Flow collections} are easily readable by a parsing engine, since they contain explicit start and end symbols. Humans on the other hand have problems with complex flow collections~\cite{connor2018flowstyle}, especially if they are not formatted properly. For an example, please take a look at the \glstext{YAML} documents in Figure~\ref{fig:block_collection} and Figure~\ref{fig:flow_collection}, which both save the same data. For a human it is much easier to determine the structure of the data in Figure~\ref{fig:block_collection}. Since configuration data should be easily readable by humans and the significant whitespace is more interesting from the point of a parsing engine, we decided to only support block collections in our “minimal” \glstext{YAML} subset.

\begin{figure}
  \begin{yamlcode}
    - Abbreviation:
       - At First: Yet Another Markup Language
       - Today: YAML Ain’t Markup Language
    - Current Version: YAML 1.2 (2009)
    - Superset of JSON
    - Two Styles: Flow Style (Indicator Based) and
                  Block Style (Indentation Based)
    - 3 Different Data Types:
      - Scalar:
        - "Hello World"
        - '👋 🌍'
        - 123
      - Sequence: [Text, 'Text', "Text", 123.5]
      - Map: {🔑: Value, see-no-evil monkey: 🙈}
    - Design Goal: “YAML is easily readable by humans.”
  \end{yamlcode}
  \caption{A YAML document that contains only block collections and flow scalars}
  \label{fig:block_collection}
\end{figure}

\begin{figure}
\begin{yamlcode}
  [{'Abbreviation':
    [{'At First': 'Yet Another Markup Language'},
     {'Today': 'YAML Ain’t Markup Language'}]},
   {'Current Version': 'YAML 1.2 (2009)'},
   'Superset of JSON',
   {'Two Styles': 'Flow Style (Indicator Based) and
                   Block Style (Indentation Based)'},
   {'3 Different Data Types':
     [{'Scalar': ['Hello World', '👋 🌍', 123]},
      {'Sequence': ['Text', 'Text', 'Text', 123.5]},
      {'Map': {'see-no-evil monkey':
       '🙈', '🔑': 'Value'}}]},
   {'Design Goal':
    '“YAML is easily readable by humans.”'}]
\end{yamlcode}
  \caption{The \glstext{YAML} document above stores the same data as the one in Figure~\ref{fig:block_collection}, but uses flow collections instead of block collections}
  \label{fig:flow_collection}
\end{figure}

In the end should also mention that other \glstext{YAML} subsets such as \href{https://github.com/crdoconnor/strictyaml}{StrictYAML}, and \href{https://metacpan.org/pod/YAML::Tiny}{YAML Tiny} also do not support references, and only have very limited support for tags and flow collections~\cite{connor2018strictyaml, kennedy2018yamltiny}.

\newpage
\section{Mapping Between Elektra’s Data Types and \glstext{YAML}}
\label{sec:mapping_elektra_yaml}

\begin{sloppypar}
  There are basically two more or less obvious solutions to map data between Elektra’s \cc{KeySet} structure and a \glstext{YAML} file. Since a \cc{KeySet} behaves similar to a map (see also section~“\nameref{sec:keyset}”), connecting a certain key to a certain value, we could use \glstext{YAML}’s map type directly.
\end{sloppypar}

\begin{figure}
  \centering
    \includegraphics[width=.5\textwidth]{Keys}
  \caption{An exemplary \cc{KeySet}}
  \label{fig:keys}
\end{figure}

For example, the \cc{KeySet} shown in figure~\ref{fig:keys} would then map to the following \glstext{YAML} data, if we use \code{user/yaml} as mountpoint:

\begin{yamlcode}
  bloc:
  bloc/party:
  bloc/party/little: "thoughts"
  bloc/party/silent: "alarm"
\end{yamlcode}

. As we can see the resulting \glstext{YAML} file contains quite a lot of unnecessary redundant data.

In our second solution we take the hierarchical nature of the database into account and split on each part of a key. The result of this approach is the following \glstext{YAML} file:

\begin{yamlcode}
  bloc:
    party:
      little: "thoughts"
      silent: "alarm"
\end{yamlcode}

. The second solution removed unwanted redundancy and reflects the hierarchy much better. However, the approach also has an obvious downside: What happens if we want to store a value in \code{user/yaml/bloc} or \code{user/yaml/bloc/party}? To answer this question, let us look at a tree representing the \glstext{YAML} data from above.

\begin{figure}
  \centering
  \begin{subfigure}[t]{.4\textwidth}
    \centering
    \includegraphics[height=4cm]{Tree}
    \caption{Initial representation}
    \label{fig:tree}
  \end{subfigure}
  \qquad
  \begin{subfigure}[t]{.4\textwidth}
    \centering
    \includegraphics[height=4cm]{TreeExtended}
    \caption{We add an additional value}
    \label{fig:tree_extended}
  \end{subfigure}\\
  \begin{subfigure}[t]{.4\textwidth}
    \centering
    \includegraphics[height=4cm]{TreeExtended+}
    \caption{We add an additional \cc{Key} containing a value}
  \end{subfigure}
  \quad
  \begin{subfigure}[t]{.4\textwidth}
    \centering
    \includegraphics[height=4cm]{TreeExtended++}
    \caption{The new \cc{Key} overwrites the value of the node \code{block}}
    \label{fig:tree_extended++}
  \end{subfigure}
  \caption{The tree-like representation of \glstext{YAML} data shows the problem with adding non-leaf values}
\end{figure}

As we can see in Figure~\ref{fig:tree} the only nodes that store values are the leaves of the tree. Let us assume we also want to store the value \code{chain} in \code{user/yaml/bloc}. Figure~\ref{fig:tree_extended} shows the resulting tree. We could now save \code{chain} as a map key inside \code{bloc}:

\begin{yamlcode}
  bloc:
    chain:
    party:
      little: "thoughts"
      silent: "alarm"
\end{yamlcode}

. However using this approach we are unable to differentiate between the name and the value of a \cc{Key}. For example, if we add a new \cc{Key} with the name \code{user/yaml/bloc/chain} it would just overwrite the value of \code{user/yaml/bloc} (see Figure~\ref{fig:tree_extended++}).

Another option to fix our problem would be to use \glstext{YAML}’s list type, and to store the value of a \cc{Key} and the data below the \cc{Key} as first and second element of the list:

\begin{yamlcode}
  bloc:
    - chain                 # First element stores value
    - party:                # Second element stores data below
      -                     # `user/yaml/bloc/party` contains no value
      - little: "thoughts"
        silent: "alarm"
\end{yamlcode}

. However, this format is quite complicated. If we add support for Elektra’s array type – mapping arrays to \glstext{YAML} lists – the situation is even worse.

To solve the problem we use another approach. We reserve the name \yaml{___dirdata} to save values in non-leaf nodes. The code below shows the mapping of our example data:

\begin{yamlcode}
  bloc:
    ___dirdata: "chain"
    party:
      little: "thoughts"
      silent: "alarm"
\end{yamlcode}

. Since we reserved the name \yaml{___dirdata} the value below this key will always be a leaf of the tree.

\subsection{Mapping Arrays}

Since Elektra’s array type and \glstext{YAML} sequences are similar, we want to map between these data types.

\begin{figure}
  \centering
    \includegraphics[width=.6\textwidth]{ArrayValue}
  \caption{The \cc{KeySet} above describes an array containing three elements}
  \label{fig:array_value}
\end{figure}

If we use this approach, then the \cc{KeySet} shown in Figure~\ref{fig:array_value} would result in the \glstext{YAML} data:

\begin{yamlcode}
  array:
    - "First Element"
    - "Second Element"
    - "Third Element"
\end{yamlcode}

. We are left with the problem, where to save the data of the \emph{parent} \cc{Key} of the array elements \code{user/array}. We can not use the same approach as before:

\begin{yamlcode}
  array:
    ___dirdata: "Array Value"
    - "First Element"
    - "Second Element"
    - "Third Element"
\end{yamlcode}

, since the result would be a \glstext{YAML} node that is neither list nor map. To fix this problem we decided to convert the \yaml{___dirdata} node to a list element:

\begin{yamlcode}
  array:
    - "___dirdata: Array Value"
    - "First Element"
    - "Second Element"
    - "Third Element"
\end{yamlcode}

. This approach produces valid \glstext{YAML} data and allows us to distinguish between array parents that store values and parents that do not, by checking the first array element for the value prefix \yaml{___dirdata:}.

\chapter{Implementation}

\section{Parsers}

\subsection{Recursive Descent Parser}

The first \href{https://github.com/ElektraInitiative/libelektra/commit/3d2d4644cb08e83f0b3305b8aeae546ada52dfe7}{YAML plugin} developed in the course of this thesis used a handwritten recursive descent parser. As already described in “\nameref{sec:state_of_the_art}” this technique is quite popular, since there is a natural correspondence between code and grammar rules. Table~\ref{tab:recursive_descent_correspondence} shows the correspondence between \gls{ABNF} grammar rules and matching C like pseudo-code.

\begin{table}[H]
  \begin{center}
    \begin{tabular}{llp{0.48\textwidth}}
      \toprule
      Grammar & Example & Code\\
      \midrule

      Terminal
      &
      \texttt{{\color{color02} a}
              {\color{color03} \textbf{=}}
              {\color{color07} "a"}}
      &
        \begin{tabminted}[autogobble]{c}
          bool a() {
            bool match = getc(file) == 'a';
            if (!match) putc(file);
            return match;
          }
        \end{tabminted}
      \\

      Sequence
      &
      \texttt{{\color{color02} seq}
              {\color{color03} \textbf{=}}
              {\color{color02} rule1 rule2}}
      &
        \begin{tabminted}[autogobble]{c}
          bool seq() {
            return rule1() && rule2();
          }
        \end{tabminted}
      \\

      Alternative
      &
      \texttt{{\color{color02} seq}
              {\color{color03} \textbf{=}}
              {\color{color02} rule1}
              {\color{color03} /}
              {\color{color02} rule2}}
      &
        \begin{tabminted}[autogobble]{c}
          bool alt() {
            return rule1() || rule2();
          }
        \end{tabminted}
      \\

      \bottomrule
    \end{tabular}
  \end{center}
  \caption{Correspondence between grammar rules and code in a recursive descent parser}
  \label{tab:recursive_descent_correspondence}
\end{table}

\newpage
While Table~\ref{tab:recursive_descent_correspondence} suggest writing a recursive descent parser is trivial, there are many problems that the code above does not take into account:

\begin{description}
  \item [Recognizer Only:] The pseudo-code only implements a recognizer for the language. At the end of the parsing process we only know if the input is part of the language produced by the given grammar or not. Usually we want to \emph{build a data structure}, in our case a \cc{KeySet}, from the given input.

  \item [Error Handling:] The code does not contain any error handling. If a given input contains errors, then the creator or editor wants to know \emph{where these errors occurred}. Otherwise she or he has to check the whole input.

  \item [Left Recursion:] If we translate a left recursive rules such as \code{\color{color02} rule1 \color{color03} \textbf{=} \color{color02} rule1 \color{color03} / \color{color02} rule2} using the correspondences given in Table~\ref{tab:recursive_descent_correspondence}, then the resulting code would never terminate. This is the case, since \code{\color{color02} rule1} calls \code{{\color{color02} rule1}}, which then calls \code{\color{color02} rule1}, and so on an so forth (infinite recursion).
\end{description}

All of the problems above apply regardless of the programming language of the parsing code. Since we implemented the recursive descent parser in C, another issue is the error handling in C itself. The language does not provide a native exception handling mechanism. We therefore used the return value to also transfer the error information between functions. This approach is quite cumbersome, since it basically means that we have to check for an error after each function call.

We used C macros to minimize the code overhead and complexity caused by the error handling. Still, for a very small part of the YAML syntax described in the ABNF grammar of Figure~\ref{cod:abnf_recurive_descent} the parser contained about 374 lines of code (counted with \code{cloc} version 1.72). While smaller feature additions, such as going from support of one key-value pair to multiple key-value pairs \href{https://github.com/ElektraInitiative/libelektra/commit/17aa7a6ea5d9261287104213dcba67f4d0a0fcbc}{were quite straightforward} (\textcolor{Green}{14 additions}, \textcolor{Red}{1 deletion}), other modifications, such as supporting block styles would take considerable more effort.

Since the first steps with a hand-written recursive descent parser showed that this approach takes considerable effort we decided against extending the first prototype. Instead we chose to use an already existing hand-written YAML parser.

\begin{figure}[htbp]
  \centering
  \begin{code-boxed}
    {\small
{\color{color02} NEL} {\color{color03} \textbf{=}} {\color{color04} \%x85}{\color{color05} \\}{\color{color02} WSLF}
{\color{color03} \textbf{=}} {\color{color04} \textbf{WSP}} {\color{color03} \textbf{/}}
{\color{color04} \textbf{LF}}{\color{color05} \\\\}{\color{color06} ; Printable
characters from C0 set\\}{\color{color02} printable} {\color{color03} \textbf{=}}
{\color{color04} \textbf{HTAB}} {\color{color03} \textbf{/}} {\color{color04} \textbf{LF}}
{\color{color03} \textbf{/}} {\color{color04} \textbf{CR}}{\color{color05} \\}{\color{color06} ;
Printable ASCII\\}{\color{color02} printable} {\color{color03} \textbf{=/}} {\color{color04} \%x20}{\color{color03} \textbf{-}}{\color{color04} 7e}{\color{color05} \\}{\color{color06} ;
Next Line from C1 set\\}{\color{color02} printable} {\color{color03} \textbf{=/}}
{\color{color02} NEL}{\color{color05} \\}{\color{color06} ; Characters after C1
set -- Surrogate pairs\\}{\color{color02} printable} {\color{color03} \textbf{=/}}
{\color{color04} \%xa0}{\color{color03} \textbf{-}}{\color{color04} d7ff}{\color{color05} \\}{\color{color06} ;
Private use characters -- Replacement character\\}{\color{color02} printable}
{\color{color03} \textbf{=/}} {\color{color04} \%xe00}{\color{color03} \textbf{-}}{\color{color04} fffd}{\color{color05} \\}{\color{color06} ;
All Unicode Character Starting from the Supplementary Multilingual Plain\\}{\color{color02} printable}
{\color{color03} \textbf{=/}} {\color{color04} \%x10000}{\color{color03} \textbf{-}}{\color{color04} 10ffff}{\color{color05} \\\\}{\color{color02} pairs}
{\color{color03} \textbf{=}} {\color{color03} \textbf{*}}{\color{color02} WSLF}
{\color{color07} \texttt{"}\{\texttt{"}} {\color{color02} pair} {\color{color02} optionalAdditionalPairs}
{\color{color07} \texttt{"}\}\texttt{"}} {\color{color03} \textbf{*}}{\color{color02} WSLF}{\color{color05} \\}{\color{color02} optionalAdditionalPairs}
{\color{color03} \textbf{=}} {\color{color03} \textbf{*(}}{\color{color07} \texttt{"},\texttt{"}}
{\color{color02} pair}{\color{color03} \textbf{)}}{\color{color05} \\}{\color{color02} pair}
{\color{color03} \textbf{=}} {\color{color02} key} {\color{color07} \texttt{"}:\texttt{"}}
{\color{color02} value}{\color{color05} \\}{\color{color02} key} {\color{color03} \textbf{=}}
{\color{color02} doubleQuotedSpace}{\color{color05} \\}{\color{color02} value}
{\color{color03} \textbf{=}} {\color{color02} doubleQuotedSpace}{\color{color05} \\}{\color{color02} doubleQuotedSpace}
{\color{color03} \textbf{=}} {\color{color03} \textbf{*}}{\color{color02} WSLF}
{\color{color02} doubleQuoted} {\color{color03} \textbf{*}}{\color{color02} WSLF}{\color{color05} \\}{\color{color02} doubleQuoted}
{\color{color03} \textbf{=}} {\color{color04} \textbf{DQUOTE}} {\color{color02} content}
{\color{color04} \textbf{DQUOTE}}{\color{color05} \\}{\color{color02} content}
{\color{color03} \textbf{=}} {\color{color03} \textbf{*}}{\color{color02} printable}
}

  \end{code-boxed}
  \caption{ABNF grammar for a very small regular subset of YAML}
  \label{cod:abnf_recurive_descent}
\end{figure}

The official \href{http://yaml.org}{YAML website} prominently lists known YAML parsers. Since we decided to use C or C++ as programming language – to improve the comparability of the parsers – we are left with three basic options:

\begin{itemize}
  \item Syck (YAML 1.0)
  \item \href{https://github.com/yaml/libyaml}{libyaml} (YAML 1.1)
  \item \href{https://github.com/jbeder/yaml-cpp}{yaml-cpp} (YAML 1.2)
\end{itemize}

. Out of these options Syck is not actively maintained any more. This leaves only \code{libyaml} and \code{yaml-cpp}. We decided to use \code{yaml-cpp}, since it supports the latest version of YAML. With the help of the library we added the first plugin with full YAML support called \href{https://www.libelektra.org/plugins/yamlcpp}{YAML CPP} to Elektra.

\section{Additional Plugins}

While most of the problems of adding a YAML storage plugin deal with the parsing process itself, there are other issues we can handle using additional plugins. Elektra’s plugin system allows us to use multiple plugins in conjunction as part of a so-called \emph{backend} (see also section~“\nameref{sec:plugins}”).

\subsection{Base64}
\label{sec:base64}

One of the first plugins we used to improve the YAML support of Elektra was the \href{https://www.libelektra.org/plugins/base64}{Base64} plugin of Peter Nirschl. The plugin en- and decodes binary values using the Base64 algorithm~\cite{josefsson2006base16}.

Since Elektra supports values containing binary data, we can use the Base64 plugin to encode this data and store it using ASCII values in a YAML file. However, the plugin used a common prefix to mark base64-encoded data. For example, if we want to store the decimal numbers 104 (0x68) and 105 (0x69), then the plugin would encode this values as \code{aGk=} and add the prefix \code{@BASE64}. The resulting value would then be \yaml{"@BASE64aGk="}. In YAML a value should not contain a prefix though. Instead YAML marks base64 encoded data with the tag (data type) \yaml{!!binary}. We therefore need to store the two values above as \yaml{!!binary "aGk="} in a YAML file. For this purpose we added a new mode to the Base64 plugin.

The new \emph{meta mode} uses metadata to mark a key-value pair that contains a base64-encoded value. Instead of a prefix Base64 adds a meta-key \code{type} with the value \code{binary}. Figure~\ref{fig:Figures_Base64} shows an example, where Elektra uses the Base64 plugin to encode and decode the bytes 0x68 and 0x69 (code points for the ASCII string \yaml{hi}).

\begin{figure}
  \centering
    \includegraphics[width=\textwidth]{Figures/Base64.pdf}
  \caption{The Base64 plugin decodes and encodes binary data}
  \label{fig:Figures_Base64}
\end{figure}

\subsection{Directory Value}

We already described the problem of storing a value in a non-leaf (directory) \cc{Key} in the Section~“\nameref{sec:mapping_elektra_yaml}”. Since the problem is independent of the parser engine and also relevant to other plugins, we implemented the functionality in a plugin named Directory Value.

The Directory Value plugin adds an additional \cc{Key} with the prefix \yaml{___dirdata} for every non-array \cc{Key} that has children and contains a value in the \code{set} direction (position \code{preset}). For example, for the \cc{KeySet} shown in Figure~\ref{fig:KeySetLarge}, the plugin adds the \cc{Key}

\begin{itemize}
  \item \code{user/yaml/bloc/\_\_\_dirdata} and
  \item \code{user/yaml/bloc/party/\_\_\_dirdata}
\end{itemize}

. The plugin then moves the data stored in the parent \cc{Key} to the newly created \cc{Key}.

\begin{figure}[H]
  \centering
  \begin{subfigure}[t]{.4\textwidth}
    \includegraphics[width=\linewidth]{KeySetLarge}
    \caption{We use the \cc{KeySet} above as input for the Directory Value plugin at the \code{preset} position.}
    \label{fig:KeySetLarge}
  \end{subfigure}
  \qquad
  \begin{subfigure}[t]{.48\textwidth}
    \includegraphics[width=\linewidth]{KeySetLargeExtended}
    \caption{The \cc{KeySet} above shows the result of the conversion at the \code{preset} position.}
    \label{fig:KeySetLargeExtended}
  \end{subfigure}
  \caption{The Directory Value plugin adds data at the position \code{preset} (\ref{fig:KeySetLarge}) and then restores the old data (\ref{fig:KeySetLargeExtended}) at the position \code{postget}.}
\end{figure}

In addition the plugin insert a new \cc{Key} for every array parent that stores a (non-binary) value, at the first position of the array. In our example, the plugin adds a new \cc{Key} with the value \code{\_\_\_dirdata: Array Value} at the first position of \code{user/yaml/array}, and increases the index of all other array elements by one.

Figure~\ref{fig:KeySetLargeExtended} shows the \cc{KeySet} after the whole conversion at the position \code{preset}. This \cc{KeySet} is also the input for the Directory Value plugin at the position \code{postget}

\chapter{Comparison}

In the evaluation phase of the work we concentrated on the criteria listed below.

\begin{description}

  \item[Run-time Performance]~\\[0.1cm]
  We used each of the parsing algorithms from the implementation phase and executed them using different input files. In each instance we measured the time it took to finish the parsing process. We repeated this multiple times for each combination of parsing technique and file, so we could determine median execution times. At the end of the phase we determined the fastest parsing code and therefore also answered~\Cref{que:speed}:

  \speed*

  \item[Memory Usage]~\\[0.1cm]
  We determined the memory usage of the different parsing implementation using \href{http://valgrind.org}{Valgrind's} heap profiler \href{http://valgrind.org/docs/manual/ms-manual.html}{Massif}.

  \item[Code Size]~\\[0.1cm]
  We counted the lines of the different implementations using the tool \href{https://github.com/AlDanial/cloc}{cloc}.

  \item[Code Complexity]~\\[0.1cm]
  For the measurement of the code complexity we determined
  \begin{itemize}
    \item the cyclomatic complexity~\cite{mccabe1976complexity}, and
    \item the Halstead complexity measures~\cite{halstead1977elements}
  \end{itemize}
  of the generated code.

  \item[Ease of Extensibility and Composability]~\\[0.1cm]
  Since there is no common way to measure either of this attributes we only looked into them from a more informal point of view. In this part of the evaluation we describe some of the features we found that hinder or enhance the extensibility and composability of the used parsing methods. At this stage we also answer~\Cref{que:closeness}:

  \closeness*

  , and argue why certain parsing libraries allow us to stay closer to the definition of the configuration language.

  \item[Error Reporting]~\\[0.1cm]
  For this subtask we created YAML input files, that contain certain errors. We then looked at what error information the parsing engines provide, and how well this information describes the given error.

  \item[Security Problems]~\\[0.1cm]
  We used a \href{https://en.wikipedia.org/wiki/Fuzzing}{fuzzer} (\href{http://lcamtuf.coredump.cx/afl}{american fuzzy lop}) to test the quality of the generated parsing code.

\end{description}

\section{Setup}

In the following section we list the hard- and software configuration we used for the comparison.

\subsection{Hardware}

For all of the tests we used the hardware described in Table~\ref{table:benchmark_hardware}.

\begin{table}[H]
  \caption{Hardware Setup}
  \label{table:benchmark_hardware}
  \centering
  \begin{tabular}{ll}
\toprule
\multicolumn{2}{c}{MacBook Pro (Retina, 15-inch, Late 2013)}\\
\midrule
\textbf{Component} & \textbf{Description}\\
\midrule
               CPU &            i7-4960HQ\\
                   &              2.6 GHz\\
                   &        6 MB L3 Cache\\
                   &      128 MB L4 cache\\
               RAM &                16 GB\\
                   &        1600 MHz DDR3\\
                HD &    Apple SSD SM1024F\\
                   &                 1 TB\\
\bottomrule
  \end{tabular}
\end{table}

\subsection{Software}

Table~\ref{table:benchmark_software} shows the overall software setup for the benchmarks. We tested the performance both on macOS and Linux. For the Linux setup we used the Mac version of Docker.

\begin{table}[H]
    \newcommand{\YAEP}[0]{{\href{https://github.com/vnmakarov/yaep/commit/550de4cc5600d5f6109c7ebcfbacec51bf80d8d3}{YAEP 550de4cc}}}
    \caption{Software Setup}
    \label{table:benchmark_software}
    \begin{subtable}[t]{.5\linewidth}
      \centering
        \caption{Mac Setup}
        \label{table:benchmark_mac}
        \begin{tabular}{ll}
\toprule
\textbf{Component} & \textbf{Description}\\
\midrule
                OS &        macOS 10.14.4\\
                   &                     \\
\midrule
          Compiler &             LLVM 8.0\\
        Generators &          ANTLR 4.7.2\\
                   &          Bison 3.3.2\\
         Libraries &       yaml-cpp 0.6.2\\
                   &                \YAEP\\
                   &          PEGTL 2.7.1\\
    Other Software &         CMake 3.14.3\\
                   &          Ninja 1.9.0\\
                   &         Xcode 10.2.1\\
\bottomrule
        \end{tabular}
    \end{subtable}
    \begin{subtable}[t]{.5\linewidth}
      \centering
        \caption{Linux Setup}
        \label{table:benchmark_docker}
        \begin{tabular}{ll}
\toprule
\textbf{Component} &      \textbf{Description}\\
\midrule
            Docker &    18.09.2, build 6247962\\
        Base Image & Debian sid (sid-20190326)\\
\midrule
          Compiler &                 GCC 8.3.0\\
        Generators &               ANTLR 4.7.2\\
                   &               Bison 3.3.2\\
         Libraries &            yaml-cpp 0.6.2\\
                   &                     \YAEP\\
                   &               PEGTL 2.8.0\\
    Other Software &              CMake 3.13.4\\
                   &               Ninja 1.8.2\\
\bottomrule
        \end{tabular}
    \end{subtable}
\end{table}

\section{Run-Time Performance}

\subsection{Profiling}

\section{Error Reporting}

\emph{Error handling} can be grouped into three dependent stages~\cite{ruefenacht2016error}:

\begin{enumerate}
  \item error detection
  \item error recovery, and
  \item error correction
\end{enumerate}

. We are mainly concerned with error detection and error recovery, since error correction is generally not possible without the possibility of fixing errors incorrectly. These errors can be disastrous in case an important configuration value, such as “radiation intensity” is set incorrectly as a result of error correction.

While there exist techniques to enhance error reporting by using external tools or modifying a parser engine~\cite{jeffery2003generating, cox2010errors}, we will only consider built-in solutions or slight modifications to a grammar. We do this, since extending a parser engine is out of scope of the thesis and elaborate extensions would also make the comparison concerning error reporting unfair.

\subsection{Initial Erroneous Input}

Listing \ref{lst:list_element_outside} shows the erroneous YAML data we used initially to compare the error reporting capabilities. Listing~\ref{lst:list_element_inside} and \ref{lst:list_element_removed} show two solutions to fix the problematic part of the YAML document.

\begin{listing}
  \begin{code-boxed}
    \inputminted[linenos]{yaml}{Data/Errors/list_element_outside.yaml}
  \end{code-boxed}
  \caption{The indentation of the sequence item \yaml{- element 2} is incorrect in the code above. One of the most obvious solutions to fix the syntax error would be to add a single space character right before \yaml{- element 2} (see Listing~\ref{lst:list_element_inside}). Another solution is to remove \yaml{- element 2} altogether (see Listing~\ref{lst:list_element_removed}).}
  \label{lst:list_element_outside}
\end{listing}

\begin{listing}
  \begin{code-boxed}
    \inputminted[linenos]{yaml}{Data/Correct/list_element_inside.yaml}
  \end{code-boxed}
  \caption{Usually a person would fix the error shown in Listing~\ref{lst:list_element_outside} by adding an indentation character before the sequence item \yaml{- element 2}.}
  \label{lst:list_element_inside}
\end{listing}

\begin{listing}
  \begin{code-boxed}
    \inputminted[linenos]{yaml}{Data/Correct/list_element_removed.yaml}
  \end{code-boxed}
  \caption{One of the easiest solutions to fix the code in Listing~\ref{lst:list_element_outside} for a computer program is to remove \yaml{- element 2}.}
  \label{lst:list_element_removed}
\end{listing}

\subsection{Basic Error Messages}

We started the comparison by listing the basic error messages for the YAML plugins. These messages contain the error location and the auto-generated error message by the parsing engines. For the sake of brevity we removed some uninteresting data such as the filename of the parsed file.

\begin{table}
  \caption{Basic error messages}
  \label{tab:error_messages_list_element_outside}
  \centering
  \begin{tabular}{llp{10cm}}
    \toprule
    \textbf{Plugin} & \textbf{Parser} & \textbf{Message}\\
    \midrule
    YAML CPP &
    yaml-cpp &
    yaml-cpp: error at line 3, column 1: end of map not found\\

    Yan LR &
    ANTLR &
    3:1: mismatched input \textquotesingle- \textquotesingle\ expecting BLOCK\_END\\

    YAMBi &
    Bison &
    3:1: syntax error, unexpected ELEMENT, \newline
    expecting KEY or BLOCK\_END\\

    YAwn &
    YAEP &
    3:1: Syntax error on token number 9: \newline
    “<Token, ELEMENT, -, 3:1–3:2>”\\

    YAy PEG &
    PEGTL &
    3:0(18): parse error matching tao::yaypeg::eof\\
    \bottomrule
  \end{tabular}
\end{table}

\subsubsection{Interpretation}

As we can see in Table~\ref{tab:error_messages_list_element_outside} all of the parsing engines report the error location for the code from Listing~\ref{lst:list_element_outside} correctly. The error messages also shows that the question, if the first position after a newline is at column 0 or 1, is still open for debate. Other than that we can see that YAMBi, YAML CPP, and Yan LR also show information about the expected element at the error position (end of a block \gls{collection} is missing). YAML CPP provides a better error message that also shows which type of end element is missing (end of map). This type of information can also be determined easily in all of the lexer-based parsing engine plugins (Yan LR, YAMBi, YAwn). We modified them accordingly. Table~\ref{tab:error_messages_improved_list_element_outside} shows the slightly improved error messages, highlighting the updated part of the text.

\begin{table}
  \caption{Slightly improved error messages}
  \label{tab:error_messages_improved_list_element_outside}
  \centering
  \begin{tabular}{llp{10cm}}
    \toprule
    \textbf{Plugin} & \textbf{Parser} & \textbf{Message}\\
    \midrule
    YAML CPP &
    yaml-cpp &
    yaml-cpp: error at line 3, column 1: end of map not found\\

    Yan LR &
    ANTLR &
    3:1: mismatched input \textquotesingle- \textquotesingle\ expecting \textbf{MAP\_END}\\

    YAMBi &
    Bison &
    3:1: syntax error, unexpected ELEMENT, \newline
    expecting \textbf{MAP\_END} or KEY\\

    YAwn &
    YAEP &
    3:1: Syntax error on token number 9: \newline
    “<Token, ELEMENT, -, 3:1–3:2>”\\

    YAy PEG &
    PEGTL &
    3:0(18): parse error matching tao::yaypeg::eof\\
    \bottomrule
  \end{tabular}
\end{table}

After the slight modifications to the YAML parser plugins we decided to take a closer look at the error handling capabilities of each of the parsing engines on their own in the next sections.

\subsection{ANTLR}

ANTLR uses an error listener class that provides a callback method that includes access to

\begin{itemize}
  \item the location,
  \item the offending symbol,
  \item the used recognizer class,
  \item the thrown exception, and
  \item the default error message
\end{itemize}

for each detected error. As we already saw in Table~\ref{tab:error_messages_list_element_outside}, the default error message provided by ANTLR usually describes errors already well. For the initial version of the \href{http://libelektra.org/plugins/yanlr}{Yan LR plugin}, we only stored the last error message reported by ANTLR. Since ANTLR uses methods such as token deletion and insertion to keep parsing a file, even if it contains multiple errors~\cite{parr2013definitive}, the last error message usually will not provide the the most obvious information on how to fix an error.

\begin{listing}
  \begin{minted}[autogobble, linenos]{yaml}
    key: - element 1
      - element 2 # Incorrect Indentation!
  \end{minted}
  \caption{The indentation of the sequence element \yaml{- element 2} is incorrect in the code above.}
  \label{lst:incorrect_indentation}
\end{listing}

For example, for the input shown in Listing~\ref{lst:incorrect_indentation} the parser produced the following error output:

\begin{textcode}
  2:37: extraneous input 'MAP END' expecting {STREAM_END, COMMENT}
\end{textcode}

. To fix this defect in the Yan LR plugin, we stored all error messages which resulted in the better error report:

\begin{textcode}
  2:1: mismatched input '- ' expecting MAP_END
  2:37: extraneous input 'MAP END' expecting STREAM_END
\end{textcode}

. You might also notice that in the error report \code{COMMENT} is missing from the list of expected tokens. This difference is the result of an \href{https://github.com/ElektraInitiative/libelektra/commit/0fe4953}{ambiguity in the ANTLR grammar we fixed}.

One of the more recent improvements in error messages of modern compilers such as Clang and GCC is the ability to highlight erroneous input. We also implemented this error reporting mechanism based on the Java code in \citetitle[page 158]{parr2013definitive}. The text:

\begin{textcode}
2:1: mismatched input '- ' expecting MAP_END
     - element 2 # Incorrect Indentation!
     ^^
2:37: extraneous input 'MAP END' expecting STREAM_END
      - element 2 # Incorrect Indentation!
                                          ^
\end{textcode}

shows the improved error message for Listing~\ref{lst:incorrect_indentation}. One thing that this error report still lacks is a more human friendly representation of the tokens. Someone with limited knowledge of the YAML specification and Yan LR’s lexer code will probably not know what \code{MAP\_END}, \code{MAP END} and \code{STREAM\_END} mean. One option to improve this situation is to replace the text used by the lexer (\code{MAP END}) and the parser (\code{MAP\_END}, \code{MAP END}). The update of the relevant lexer code is trivial, since we can create tokens containing arbitrary text. For the parser code generated by ANLTR, we used a script that uses regular expressions to replace the relevant strings such as \cc{"MAP_END"} and \cc{"STREAM_END"}. After this update the error report for the YAML data in Listing~\ref{lst:incorrect_indentation} looks like this:

\begin{textcode}
2:1: mismatched input '- ' expecting end of map
     - element 2 # Incorrect Indentation!
     ^^
2:37: extraneous input 'end of map' expecting end of document
      - element 2 # Incorrect Indentation!
                                          ^
\end{textcode}

.

\subsection{Bison}

The first step we used to improve the error messages of the Bison parser was to define alternative names for the tokens, just like we did for Yan LR. Bison supports this feature directly, which meant we did not have to write a script to replace the symbols in the generated parser code. After this update the error message from Table~\ref{tab:error_messages_improved_list_element_outside} changed from:

\begin{textcode}
  3:1: syntax error, unexpected ELEMENT, expecting MAP_END or KEY
\end{textcode}

to

\begin{textcode}
  3:1: syntax error, unexpected element, expecting end of map or key
\end{textcode}

.

We then looked into the error recovery capabilities of Bison. Unlike ANTLR the generated parser does not do error recovery by default, but rather exits on the first error. To improve the error behavior Bison offers the possibility to add the predefined \code{error} token to a grammar. Every time the Bison parser encounters an error it will produce this token~\cite{donnelly2019bison}. We modified the grammar to allow errors inside YAML maps and sequences:

\begin{ccode}
  pairs : pair
        | pairs pair
        | pairs error /* Allow errors after key-value pairs */
        ;

  elements : element
           | elements element
           /* Allow errors after elements of a sequence */
           | elements error
           ;
\end{ccode}

. This way the parser is able to report multiple syntax errors.

\begin{listing}
  \begin{minted}[autogobble, linenos]{yaml}
    key 1: - element 1
     - element 2
    key 2: scalar
           - element 3
  \end{minted}
  \caption{The indentation of the sequence item \yaml{- element 2} is incorrect in the code above. Another error is that the value of \yaml{key 2} can not be both a scalar (\yaml{scalar}) and a sequence (containing \yaml{- element 3}).}
  \label{lst:incorrect_indentation_element_without_sequence}
\end{listing}

For example, for the input shown in Listing~\ref{lst:incorrect_indentation_element_without_sequence} the parser produces an error message that looks like this:

\begin{textcode}
2:2: unexpected start of sequence, expecting end of map or key
      - element 2
      ^
4:8: unexpected start of sequence, expecting end of map or key
            - element 3
            ^
\end{textcode}

. As you can see above, we also added the erroneous input to the error message, just like we did in the Yan LR plugin.

One option to improve the error output of the Bison parser we did not implement is described in the paper \citetitle{jeffery2003generating} by \citeauthor{jeffery2003generating}~\cite{jeffery2003generating}. The technique looks promising, and was also added to the old Bison parser\footnote{The developers of Go switched to a hand-written parser written in Go in 2016~\cite{pike2017reddit, go2016release}.} for the programming language Go~\cite{cox2010errors}. However, the technique is not directly supported by Bison and hence requires changes to the parsing code, that are out of the scope of this thesis.

\subsection{YAEP}

Just as Bison, YAEP also requires that you add error tokens to the grammar to specify locations for error recovery. We therefore defined the same error recovery locations inside sequences and maps, as we did with Bison. The other updates were quite similar too: We improved the name of tokens inside error messages and added the erroneous input to the error message.

After all these changes the output for the YAML data from Listing~\ref{lst:incorrect_indentation_element_without_sequence} looks very similar to the one produced by YAMBi:

\begin{textcode}
2:2: Syntax error on input “start of sequence”
      - element 2
      ^
4:8: Syntax error on input “start of sequence”
            - element 3
            ^
\end{textcode}

. The only thing missing is the information about the expected type of token.

\subsection{PEGTL}

LL and LR parsers read the input deterministically from left to right. They therefore report the first position, where the parsed input is not part of the language described by the grammar anymore. This behavior is also known as (longest) correct/viable prefix property~\cite{sippu1990parsing, ruefenacht2016error, maidl2016129}.

\Gls{PEG} parsers do not have this property. The reason behind this is that \glspl{PEG} use backtracking, and therefore obfuscate error locations~\cite{ruefenacht2016error}. In his master thesis~\cite{ford2002packrat} \citeauthor{ford2002packrat} describes one option to produce meaningful error messages. His parser records all parsing results and uses the one that matched the farthest to the right in the input for error messages. In \citetitle{maidl2016129}~\cite{maidl2016129} \citeauthor{maidl2016129} show that this error strategy can also be added to every \gls{PEG} library that supports semantic actions. They also introduce a form of error reporting, inspired by the excepting handling mechanism of programming languages, based on grammar annotations called labeled failures~\cite{maidl2016129}.

PEGTL does neither implement the error handling strategy described by \citeauthor{ford2002packrat}~\cite{ford2002packrat}, nor labelled failures~\cite{maidl2016129}. Instead the library offers a grammar rule called \cpp{must}, which states that a certain rule, specified as template argument, has to match at a given position or an error will be raised. We can customize the code executed for a given \cpp{must} rule according to this template argument. Effectively this strategy allows us to specify different error messages for each expected but unmatched rule.

As we described in the section “\nameref{sec:peg_parser}”, we tried to keep the grammar of our PEG Parser plugin \href{https://libelektra.org/plugins/yaypeg}{YAy PEG} close to the grammar of the \href{http://yaml.org/spec/1.2/spec}{YAML specification}~\cite{ben2009yaml}. This also meant, that the grammar contained only a single \cpp{must} rule, that makes sure that the grammar matched the whole input:

\begin{cppcode}
  struct yaml : if_must<l_yaml_stream, eof> {};
\end{cppcode}

. The code above also explains the initial version of the error message shown in Table~\ref{tab:error_messages_improved_list_element_outside}:

\begin{textcode}
  3:0(18): parse error matching tao::yaypeg::eof
\end{textcode}

, which tells us, that the parser was unable to match the expected “end of file” in line 3 of the input. We customized the error message above to show a more user friendly text:

\begin{textcode}
  3:0: Incomplete document, expected “end of file”
       - element 2
       ^
\end{textcode}

. As you can see we also added the erroneous input to the message, just as we did for the other parsing engines.

The same single error message, regardless of the error, is not helpful. For good error reporting we need to add other \cpp{must} rules. However, adding failure points (\cpp{must} rules) changes the behavior of the grammar and might even cause the parser to fail on valid input. To minimize the probability of incorrect grammar changes we only added a few rules for situation we were sure that the remainder of a grammar rule had to match. For example, when the parser reads an unescaped single or double quote character (outside of a block scalar) at the beginning of a line or after a whitespace character, it found a quoted flow scalar. Therefore

\begin{enumerate}
  \item the text after the initial quote has to be followed by a (possibly empty) text containing only certain characters, and
  \item the last character of the flow scalar has to be an unescaped quote
\end{enumerate}

. If one of those two rules is not fulfilled, then the parser found a syntax error. After we updated the code accordingly the error message for the YAML data

\begin{yamlcode}
  "double quoted
\end{yamlcode}

looks like this:

\begin{textcode}
  1:14: Missing closing double quote or
        incorrect value inside flow scalar
        "double quoted
                      ^
\end{textcode}

. As you noticed we included both error possibilities in the error message, since reacting to both errors independently would require fundamental changes to the grammar.

\subsection{Final Error Messages}

\subsubsection{Element Outside of Sequence}

Table~\ref{tab:error_messages_final_list_element_outside} shows the final error messages for the code of Listing~\ref{lst:list_element_outside}:

\begin{code-boxed}
  \inputminted[linenos]{yaml}{Data/Errors/list_element_outside.yaml}
\end{code-boxed}

.

\begin{table}[H]
  \caption{Final error messages for the YAML code of Listing~\ref{lst:list_element_outside}}
  \label{tab:error_messages_final_list_element_outside}

  \centering
  \begin{tabular}{lp{0.8\textwidth}}
    \toprule
    Plugin & Error Messages\\
    \midrule

    \vspace{0cm}
    YAML CPP &
    \vspace{-0.36cm}
    \begin{textcode}
      error at line 3, column 1: end of map not found.
    \end{textcode}
    \\

    \vspace{0cm}
    Yan LR &
    \vspace{-0.36cm}
    \begin{textcode}
      3:1: mismatched input '- ' expecting end of map
           - element 2
           ^^
    \end{textcode}
    \\

    \vspace{0cm}
    YAMBi &
    \vspace{-0.36cm}
    \begin{textcode}
      3:1: syntax error, unexpected element,
           expecting end of map or key
           - element 2
           ^^
    \end{textcode}
    \\

    \vspace{0cm}
    YAwn &
    \vspace{-0.36cm}
    \begin{textcode}
      3:1: Syntax error on input “-”
           - element 2
           ^^
    \end{textcode}
    \\

    \vspace{0cm}
    YAy PEG &
    \vspace{-0.36cm}
    \begin{textcode}
      3:0: Incomplete document, expected “end of file”
           - element 2
           ^
    \end{textcode}
    \\

    \bottomrule

  \end{tabular}
\end{table}

\paragraph{Interpretation}

\begin{itemize}
  \item All the parsing engines report the correct error location.
  \item YAy PEG and YAwn do not tell us in which YAML node the error occurs. All the other plugins report that a map ended prematurely.
  \item The error message of YAMBi reports an additional option – besides deleting the input – to fix the error: adding a key (in the line between \yaml{- element 1} and \yaml{- element 2}).
\end{itemize}

The following list shows a ranking of the plugins according to the interpretation of the error messages above.

\begin{enumerate}
  \item YAMBi
  \item YAML CPP, Yan LR
  \item YAy PEG, YAwn
\end{enumerate}

\subsubsection{YAML Data Containing Multiple Errors}

The YAML data from Listing~\ref{lst:list_element_outside} only contains a single syntax error. To compare the error recovery capabilities of the parsing libraries, we used YAML data that contains multiple syntax errors as input (see Listing~\ref{lst:multiple_errors}).

\begin{listing}
  \begin{code-boxed}
    \inputminted[linenos]{yaml}{Data/Errors/multiple_errors.yaml}
  \end{code-boxed}
  \caption{The YAML data above contains three syntax errors that we directly describe in the comments right next to the error positions.}
  \label{lst:multiple_errors}
\end{listing}

Table~\ref{tab:error_messages_final_multiple_errors} shows the error messages of the different storage plugins for the YAML input of Listing~\ref{lst:multiple_errors}. As we can see the error output is quite different.

\begin{table}[H]
  \caption{Error messages for the YAML code of Listing~\ref{lst:multiple_errors}}
  \label{tab:error_messages_final_multiple_errors}
  \centering

  \begin{tabular}{lp{0.9\textwidth}}
    \toprule
    Plugin & Error Messages\\
    \midrule

    \vspace{0cm}
    YAML CPP &
    \vspace{-0.36cm}
    \begin{textcode}
      error at line 5, column 5: end of map not found.
    \end{textcode}
    \\

    \vspace{0cm}
    Yan LR &
    \vspace{-0.36cm}
    \begin{textcode}
      5:5: mismatched input 'element 3' expecting end of sequence
               element 3 # Missing `- `
               ^^^^^^^^^
      6:1: extraneous input 'end of sequence' expecting end of map
           key 3: "double quoted scalar"
           ^
      11:4: mismatched input 'start of map' expecting end of map
               key 6: # Not on same level as key 5

      13:1: mismatched input 'end of map' expecting end of document
            key 7: 'single quoted scalar'
            ^
    \end{textcode}
    \\

    \vspace{0cm}
    YAMBi &
    \vspace{-0.36cm}
    \begin{textcode}
      5:5: syntax error, unexpected plain scalar,
           expecting end of sequence or element
               element 3 # Missing `- `
               ^^^^^^^^^
      11:4: syntax error, unexpected start of map,
            expecting end of map or key
               key 6: # Not on same level as key 5
               ^
      13:1: syntax error, unexpected key, expecting end of document
            key 7: 'single quoted scalar'
            ^
    \end{textcode}
    \\

    \vspace{0cm}
    YAwn &
    \vspace{-0.36cm}
    \begin{textcode}
      5:5: Syntax error on input “element 3”
               element 3 # Missing `- `
               ^^^^^^^^^
      11:4: Syntax error on input “start of map”
               key 6: # Not on same level as key 5
               ^
      13:1: Syntax error on input “key”
            key 7: 'single quoted scalar'
            ^
      16:1: Syntax error on input “scalar”
            scalar # Not a key
            ^^^^^^
    \end{textcode}
    \\

    \vspace{0cm}
    YAy PEG &
    \vspace{-0.36cm}
    \begin{textcode}
      5:0: Incomplete document, expected “end of file”
               element 3 # Missing `- `
           ^
    \end{textcode}
    \\

    \bottomrule

  \end{tabular}

\end{table}

\paragraph{Interpretation}

\begin{itemize}

  \item \emph{YAML CPP} and \emph{YAy PEG} do not provide any error recovery.

  \item \emph{YAy PEG} shows the correct line number for the first error, but not the the correct column number. The plugin also only provides a very generic error message.

  \item All of the plugins that use error recovery (Yan LR, YAMBi, YAwn) print a spurious error messages at line 13.

  \item \emph{Yan LR} shows two error messages for the first syntax error, and one for the second syntax error. Error messages one and three describe the problematic part of the YAML data reasonably well.

  \item Compared to Yan LR, \emph{YAMBi’s} (non-spurious) error messages also describe a second option to fix the erroneous input. However, while the first error message provides a useful suggestion on how to fix the error (insertion of a sequence element), the second option in the second error messages (insertion of a key), will probably confuse anyone that does not know how YAMBi’s lexer works.

  \item \emph{YAwn} prints the same error messages as YAMBi, without the crucial information about the expected element. In addition YAwn prints a fourth error message that addresses the third syntax error.

\end{itemize}

According to the interpretation we concluded that all of the plugins with error recovery provide about the same level of useful error information. YAML CPP describes the first error reasonably well, while the error message from YAy PEG is not that useful. This leaves us with the following ranking of the error capabilities of the plugins based on the input of Listing~\ref{lst:multiple_errors}:

\begin{enumerate}
  \item Yan LR, YAMBi, Yawn
  \item YAML CPP
  \item YAy PEG
\end{enumerate}

.

\subsubsection{Conclusion}

While the parsing libraries do not produce particularly great error messages, at least the ANTLR (Yan LR) and Bison (YAMBi) plugin, provide error messages that are comparable in quality to the ones of the handwritten parsing engine (YAML CPP).

One advantage of Yan LR, YAMBi, and YAwn is that their parsers offer error recovery. They are therefore able to report multiple errors in a file. This is something that YAML CPP is currently not able to do. ANTLR offers error recovery for free, while Bison and YAEP require us to add error tokens to the grammar. This can be problematic, since these error tokens can produce conflicts in the case of Bison, and ambiguous parsing results in the case of YAEP.

The parsing plugin that showed the least useful error messages is YAy PEG. While the PEGTL offers basic error handling facilities that are able to provide good error messages for character level errors, producing good error messages for “high-level” errors would probably require a substantial amount of work.


\backmatter

% -- Glossary & References -----------------------------------------------------

\printglossaries
\begin{sloppypar}\printbibliography\end{sloppypar}

\end{document}
