\chapter{Conclusion}

% Determined promising parsing techniques according to current literature:
% → Only use directed parsing techniques, since they should be fast and powerful enough

% Determined good candidates for parsing libraries an:
% → All of them written in C/C++ to improve comparability

% Found minimal YAML subset in discussion with Elektra developers:
% → Developers preferred double quoted scalars
% → Did not implement all of the stated features like typing and references, since they are either not really applicable to Elektra (references) or require work that is out of the scope of the thesis (typing)

% Determined mapping between YAML and Elektra `keySet`:
% → Requires way to translate non-leaf values to leaf values

% Implemented plugins:
% → Problems
%   → Begin and end markers in lexer based plugins
%   → Handling dynamic context in PEG parser

% Comparison:
% → All parsing engines show linear runtime/memory usage for tested file
% → Surprisingly good performance results of Earley parser
% → Sometimes large difference between operating systems
% → ANTLR provides most support code, other plugins required reimplementation of the support code
% → ANTLR provides good error messages without requiring any changes to the grammar
% → ANTLR most promising because of:
%   → support code
%   → good error messages
%   → large community
