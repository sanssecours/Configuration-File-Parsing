\chapter{Comparison}

In the evaluation phase of the work we concentrated on the criteria listed below.

\begin{description}

  \item[Run-time Performance]~\\[0.1cm]
  We used each of the parsing algorithms from the implementation phase and executed them using different input files. In each instance we measured the time it took to finish the parsing process. We repeated this multiple times for each combination of parsing technique and file, so we could determine median execution times. At the end of the phase we determined the fastest parsing code and therefore also answered~\Cref{que:speed}:

  \speed*

  \item[Memory Usage]~\\[0.1cm]
  We determined the memory usage of the different parsing implementation using \href{http://valgrind.org}{Valgrind's} heap profiler \href{http://valgrind.org/docs/manual/ms-manual.html}{Massif}.

  \item[Code Size]~\\[0.1cm]
  We counted the lines of the different implementations using the tool \href{https://github.com/AlDanial/cloc}{cloc}.

  \item[Code Complexity]~\\[0.1cm]
  For the measurement of the code complexity we determined
  \begin{itemize}
    \item the cyclomatic complexity~\cite{mccabe1976complexity}, and
    \item the Halstead complexity measures~\cite{halstead1977elements}
  \end{itemize}
  of the generated code.

  \item[Ease of Extensibility and Composability]~\\[0.1cm]
  Since there is no common way to measure either of this attributes we only looked into them from a more informal point of view. In this part of the evaluation we describe some of the features we found that hinder or enhance the extensibility and composability of the used parsing methods. At this stage we also answer~\Cref{que:closeness}:

  \closeness*

  , and argue why certain parsing libraries allow us to stay closer to the definition of the configuration language.

  \item[Error Reporting]~\\[0.1cm]
  For this subtask we created YAML input files, that contain certain errors. We defined reference error messages that describe the problem in a way that is as user-friendly as possible. We then looked how far we can approximate the ideal messages with the different parsing methods.

  \item[Security Problems]~\\[0.1cm]
  We used a \href{https://en.wikipedia.org/wiki/Fuzzing}{fuzzer} (\href{http://lcamtuf.coredump.cx/afl}{american fuzzy lop}) to test the quality of the generated parsing code.

\end{description}

\section{Error Reporting}

\emph{Error handling} can be grouped into three dependent stages~\cite{ruefenacht2016error}:

\begin{enumerate}
  \item Error Detection
  \item Error Recovery
  \item Error Correction
\end{enumerate}

. We are mainly concerned with error detection, since error recovery and error correction are generally not possible without the possibility of fixing errors incorrectly. These errors can be disastrous in case an important configuration value, such as “radiation intensity” is set incorrectly as a result of error correction.

While there exist techniques to enhance error reporting using external tools or modifying a parser engine~\cite{jeffery2003generating, cox2010errors}, we will only consider built-in solutions or slight modifications to a grammar. We do this, since extending a parser engine is out of scope of the thesis and elaborate extensions would also make the comparison concerning error reporting unfair.

\subsection{Initial Erroneous Input}

Listing \ref{lst:list_element_outside} shows the erroneous YAML data we used initially to compare the error reporting capabilities. Listing~\ref{lst:list_element_inside} and \ref{lst:list_element_removed} show two solutions to fix the problematic part of YAML document.

\begin{listing}
  \begin{code-boxed}
    \inputminted{yaml}{Data/Errors/list_element_outside.yaml}
  \end{code-boxed}
  \caption{The indentation of the plain scalar \yaml{element 2} is incorrect in the code above. One of the most obvious solutions to fix the syntax error would be to add a single space character right before \yaml{element 2} (see Listing~\ref{lst:list_element_inside}). Another solution is to remove \yaml{element 2} altogether (see Listing~\ref{lst:list_element_removed}).}
  \label{lst:list_element_outside}
\end{listing}

\begin{listing}
  \begin{code-boxed}
    \inputminted{yaml}{Data/Correct/list_element_inside.yaml}
  \end{code-boxed}
  \caption{Usually a person would fix the error shown in Listing~\ref{lst:list_element_outside} by adding an indentation character before the plain scalar \yaml{element 2}.}
  \label{lst:list_element_inside}
\end{listing}

\begin{listing}
  \begin{code-boxed}
    \inputminted{yaml}{Data/Correct/list_element_removed.yaml}
  \end{code-boxed}
  \caption{One of the easiest solutions to fix the code in Listing~\ref{lst:list_element_outside} for a computer program is to remove \yaml{element 2}.}
  \label{lst:list_element_removed}
\end{listing}

\subsection{Basic Error Messages}

We started the comparison by listing the basic error messages for the YAML plugins. These messages contain the error location and the auto-generated error message by the parsing engines. For the sake of brevity we removed some uninteresting data such as the filename of the parsed file.

\begin{table}
  \caption{Basic Error Messages}
  \label{tab:error_messages_list_element_outside}
  \centering
  \begin{tabular}{llp{10cm}}
    \toprule
    \textbf{Plugin} & \textbf{Parser} & \textbf{Message}\\
    \midrule
    YAMBi &
    Bison &
    3:1: syntax error, unexpected ELEMENT, \newline
    expecting KEY or BLOCK\_END\\

    YAML CPP &
    yaml-cpp &
    yaml-cpp: error at line 3, column 1: end of map not found\\

    Yan LR &
    ANTLR &
    3:1: mismatched input \textquotesingle- \textquotesingle\ expecting BLOCK\_END\\
    YAwn &
    YAEP &
    3:1: Syntax error on token number 9: \newline
    “<Token, ELEMENT, -, 3:1–3:2>”\\

    YAy PEG &
    PEGTL &
    3:0(18): parse error matching tao::yaypeg::eof\\
    \bottomrule
  \end{tabular}
\end{table}

\subsubsection{Interpretation}

As we can see in Table~\ref{tab:error_messages_list_element_outside} all of the parsing engines report the error location for the code from Listing~\ref{lst:list_element_outside} correctly. The error messages also shows that the question, if the first position after a newline is at column 0 or 1, is still open for debate. Other than that we can see that YAMBi, YAML CPP, and Yan LR also show information about the expected element at the error position (end of a block \gls{collection} is missing). YAML CPP provides a better error message that also shows which type of end element is missing (end of map). This type of information can also be determined easily in all of the lexer-based parsing engine plugins (Yan LR, YAMBi, YAwn). We modified them accordingly. Table~\ref{tab:error_messages_improved_list_element_outside} shows the slightly improved error messages, highlighting the updated part of the text.

\begin{table}
  \caption{Slightly Improved Error Messages}
  \label{tab:error_messages_improved_list_element_outside}
  \centering
  \begin{tabular}{llp{10cm}}
    \toprule
    \textbf{Plugin} & \textbf{Parser} & \textbf{Message}\\
    \midrule
    YAMBi &
    Bison &
    3:1: syntax error, unexpected ELEMENT, \newline
    expecting \textbf{MAP\_END} or KEY\\

    YAML CPP &
    yaml-cpp &
    yaml-cpp: error at line 3, column 1: end of map not found\\

    Yan LR &
    ANTLR &
    3:1: mismatched input \textquotesingle- \textquotesingle\ expecting \textbf{MAP\_END}\\
    YAwn &
    YAEP &
    3:1: Syntax error on token number 9: \newline
    “<Token, ELEMENT, -, 3:1–3:2>”\\

    YAy PEG &
    PEGTL &
    3:0(18): parse error matching tao::yaypeg::eof\\
    \bottomrule
  \end{tabular}
\end{table}
