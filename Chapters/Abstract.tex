\begin{abstract}
Parsing – determining structure of input according to a language grammar – has been an important area of research in computer science for decades. One might assume that the results of this research, such as parser generators and parser libraries, are prevalent tools all programmers use when they decide to work with any computer language. However, this is not the case. Especially for custom and simple languages, such as configuration file formats, people tend to write their own parsing code, often ignoring lessons learned in the past. While this might work reasonably well for regular languages, there are many features parser generators and parsing libraries offer that developers will probably not implement in their custom parsers, such as Unicode support and error recovery. Using a library to access the structure of a language fixes some of these problems. However, those libraries do not exist for every languages and might not provide access to certain important parts of the input such as comments. To improve the current situation we looked at and compared current parsing tools. For this purpose we parse a subset of the language \glstext{YAML} and convert it into the data structures of the configuration framework Elektra. This common output structure allows a fair comparison of the parsing tools that usually produce different kind of parse trees or execute custom code as part of the parsing process.

This thesis starts with a literature research that highlights the current state of art in parsing. At the end of the literature research we chose parsing libraries and generators for the most promising parsing techniques, namely \glstext{ALL(*)}, LR, Earley parsing, \glstext{PEG} parsing, parser combinators, and bidirectional programming. After that we decided about the YAML features we wanted to implement in a discussion with some of Elektra’s developers. Then we chose the mapping between YAML and Elektra’s data structures and finally implemented the parsers for our YAML subset. In the implementation phase we skipped the parser combinator library mpc, and the bidirectional programming library Augeas. While Augeas was not powerful enough to process our YAML subset, we could not find any advantage of mpc over the conceptually similar PEG library PEGTL.

The benchmarks and comparison of various features of the parsing tools showed that for our example language \glstext{ANTLR} provided the most complete feature set and good error messages without requiring any changes to the grammar. While the parser generated by \glstext{ANTLR} was neither the fastest nor the parser with the lowest memory overhead, the benchmarks in our comparison still showed acceptable performance for our example data.
\end{abstract}
