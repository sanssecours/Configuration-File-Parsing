\begin{kurzfassung}
Parsen – das Erfassen der Struktur eines Textes an Hand einer gegebenen Grammatik – ist seit Jahrzehnten ein wichtiger Teil der Forschung in der Informatik. Man würde vermuten dass die Resultate dieser Forschung, Parsergeneratoren und Parserbibliotheken, vorherrschende Werkzeuge sind, wenn Programmierer sich entscheiden eine Computersprache zu verarbeiten. Das ist jedoch nicht der Fall. Speziell für spezialisierte und einfache Sprachen, wie sie z.B. für Konfigurationsdaten verwendet werden, entscheiden sich Programmierer oft ihren eigenen Parsercode zu schreiben und ignorieren dabei in der Vergangenheit gewonnene Erkenntnisse. Währen dieser Ansatz für reguläre Sprachen gute Ergebnisse liefern kann, gibt es viele Funktionen die Parsergeneratoren und Parserbibliotheken implementieren, die ein Entwickler üblicherweise nicht in seinen handgeschriebenen Parsercode integrieren wird, wie z.B. Unicode-Support und Weiterverarbeitung eines Textes nach einem Syntax-Fehler. Die Verwendung einer Bibliothek für eine bestimmte Computersprache kann einige dieser Problemen lösen. Allerdings existieren solche Bibliotheken nicht für jede Sprache. Vieler dieser Bibliotheken bieten außerdem keinen Zugriff auf wichtige Daten eines Textes, wie z.B. Kommentare. Um die gegenwärtige Situation zu verbessern beschäftigt sich diese Diplomarbeit mit aktuellen Parserwerkzeugen und vergleicht diese. Für diese Zweck parsen wir eine Teilmenge der Sprache \glstext{YAML} und wandeln Dokumente in dieser Sprache in die Datenstrukturen des Konfiguration-Frameworks Elektra um. Diese gemeinsame Datenstruktur erlaubt einen sachlichen Vergleich der Parserwerkzeuge die üblicherweise verschiedene Arten von Syntax-Bäumen erzeugen oder benutzerdefinierten Code während des Parse-Vorgangs ausführen.
\end{kurzfassung}
