\chapter{Approach}

Since the YAML standard is quite extensive~\cite{ben2009yaml} – the document describing the serialization language includes about 200 parameterized \gls{BNF} grammar rules – we decided to first determine a useful subset of YAML. For this purpose we discussed the format with other Elektra developers.

\section{Discussion}

In the discussion 9 participants answered questions about the usefulness of certain YAML features for \href{https://www.libelektra.org}{Elektra}, a cross-plattform configuration library. Since YAML is not trivial, we introduced the format in a \href{https://github.com/sanssecours/YAML-Presentation/releases/download/v1.0/Presentation.pdf}{presentation}. After we talked about a certain part of YAML, we answered questions the participants had about the information presented so far. Afterwards we asked the participants to fill in parts of a \href{https://github.com/sanssecours/YAML-Presentation/blob/master/Questionnaire.md}{questionnaire} about the newly introduced feature set. The questionnaire consisted of a checkbox for each feature. A checked box means that the participant considers the feature useful for Elektra, while a checkbox without a mark means the opposite.

\subsection{Participants}

All of the 9 participants were at least partially familiar with Elektra. Some also had previous experience with YAML. Seven of them listened to the presentation, while one participant was late and another one participated via eMail. The eMail participant received a copy of the presentation slides and the questionnaire.

\subsection{Results}

In the following bar charts the term “Yes” refers to a checked box for the specific feature. The term “?” means that the participant did not know enough about a part of YAML and therefore marked the checkbox for one feature, or the heading for multiple features, with a question mark. The value before the term “No” specifies the number of unchecked boxes minus the number of boxes marked with “?”.

\subsubsection{Scalars}

\paragraph{Flow Scalars}

\begin{figure}[H]
  \begin{minipage}[t]{0.48\textwidth}
    \vspace{0pt}
    \begin{bchart}[max=9, width=0.85\textwidth]
      \bcbar[text=3, value=Yes, color=orange]{3}
      \bcbar[text=6, value=No, color=Aqua]{6}
    \end{bchart}
  \end{minipage}
  \begin{minipage}[t]{0.48\textwidth}
    \vspace{0pt}
    \begin{yamlcode}
      Plain String
    \end{yamlcode}
  \end{minipage}
  \caption{Plain Flow Scalar}
\end{figure}

\begin{figure}[H]
  \begin{minipage}[t]{0.48\textwidth}
    \vspace{0pt}
    \begin{bchart}[max=9, width=0.85\textwidth]
      \bcbar[text=2, value=Yes, color=orange]{2}
      \bcbar[text=7, value=No, color=Aqua]{7}
    \end{bchart}
  \end{minipage}
  \begin{minipage}[t]{0.48\textwidth}
    \vspace{0pt}
    \begin{yamlcode}
      'Single Quoted ''String'''
    \end{yamlcode}
  \end{minipage}
  \caption{Single Quoted Flow Scalar}
\end{figure}

\begin{figure}[H]
  \begin{minipage}[t]{0.48\textwidth}
    \vspace{0pt}
    \begin{bchart}[max=9, width=0.85\textwidth]
      \bcbar[text=8, value=Yes, color=orange]{8}
      \bcbar[text=1, value=No, color=Aqua]{1}
    \end{bchart}
  \end{minipage}
  \begin{minipage}[t]{0.48\textwidth}
    \vspace{0pt}
    \begin{yamlcode}
      "Double\n Quoted\n \"String\""
    \end{yamlcode}
  \end{minipage}
  \caption{Double Quoted Flow Scalar}
\end{figure}

\paragraph{Block Scalars}

\begin{figure}[H]
  \begin{minipage}[t]{0.48\textwidth}
    \vspace{0pt}
    \begin{bchart}[max=9, width=0.85\textwidth]
      \bcbar[text=2, value=Yes, color=orange]{2}
      \bcbar[text=6, value=No, color=Aqua]{6}
      \bcbar[text=1, value=?, color=DarkTurquoise]{1}
    \end{bchart}
  \end{minipage}
  \begin{minipage}[t]{0.48\textwidth}
    \vspace{0pt}
    \begin{yamlcode}
      > # "Folded Style"
        Folded
        Style
    \end{yamlcode}
  \end{minipage}
  \caption{Folded Block Scalar}
\end{figure}

\begin{figure}[H]
  \begin{minipage}[t]{0.48\textwidth}
    \vspace{0pt}
    \begin{bchart}[max=9, width=0.85\textwidth]
      \bcbar[text=2, value=Yes, color=orange]{2}
      \bcbar[text=6, value=No, color=Aqua]{6}
      \bcbar[text=1, value=?, color=DarkTurquoise]{1}
    \end{bchart}
  \end{minipage}
  \begin{minipage}[t]{0.48\textwidth}
    \vspace{0pt}
    \begin{yamlcode}
      | # "Literal\nStyle"
        Literal
        Style
    \end{yamlcode}
  \end{minipage}
  \caption{Literal Block Scalar}
\end{figure}

\begin{figure}[H]
  \begin{minipage}[t]{0.48\textwidth}
    \vspace{0pt}
    \begin{bchart}[max=9, width=0.85\textwidth]
      \bcbar[text=1, value=Yes, color=orange]{1}
      \bcbar[text=7, value=No, color=Aqua]{7}
      \bcbar[text=1, value=?, color=DarkTurquoise]{1}
    \end{bchart}
  \end{minipage}
  \begin{minipage}[t]{0.48\textwidth}
    \vspace{0pt}
    \begin{yamlcode*}{showspaces, spacecolor = lightgray, space=·}
      >1 # "  1 Space Indentation"
         1 Space Indentation
    \end{yamlcode*}
  \end{minipage}
  \caption{Indentation Header}
\end{figure}

\begin{figure}[H]
  \begin{minipage}[t]{0.48\textwidth}
    \vspace{0pt}
    \begin{bchart}[max=9, width=0.85\textwidth]
      \bcbar[text=0, value=$\quad$Yes, color=orange]{0}
      \bcbar[text=8, value=No, color=Aqua]{8}
      \bcbar[text=1, value=?, color=DarkTurquoise]{1}
    \end{bchart}
  \end{minipage}
  \begin{minipage}[t]{0.48\textwidth}
    \vspace{0pt}
    \begin{yamlcode*}{showspaces, spacecolor = lightgray, space=·, escapeinside=||}
      >- # "No Trailing Whitespace"
         No Trailing Whitespace
        | |
        | |
      # ↑ Newlines Above Stripped
    \end{yamlcode*}
  \end{minipage}
  \caption{Chomping Header}
\end{figure}

\subsubsection{Lists}

\begin{figure}[H]
  \begin{minipage}[t]{0.48\textwidth}
    \vspace{0pt}
    \begin{bchart}[max=9, width=0.85\textwidth]
      \bcbar[text=5, value=Yes, color=orange]{5}
      \bcbar[text=4, value=No, color=Aqua]{4}
    \end{bchart}
  \end{minipage}
  \begin{minipage}[t]{0.48\textwidth}
    \vspace{0pt}
    \begin{yamlcode}
      [🍎, 🍊,
        [Sugar, Eggs, Chocolate]
      ]
    \end{yamlcode}
  \end{minipage}
  \caption{Flow Style}
\end{figure}

\begin{figure}[H]
  \begin{minipage}[t]{0.48\textwidth}
    \vspace{0pt}
    \begin{bchart}[max=9, width=0.85\textwidth]
      \bcbar[text=7, value=Yes, color=orange]{7}
      \bcbar[text=2, value=No, color=Aqua]{2}
    \end{bchart}
  \end{minipage}
  \begin{minipage}[t]{0.48\textwidth}
    \vspace{0pt}
    \begin{yamlcode}
      - 🍎
      - 🍊
      - - Sugar
        - Eggs
        - Chocolate
    \end{yamlcode}
  \end{minipage}
  \caption{Block Style}
\end{figure}

\subsubsection{Mappings}

\begin{figure}[H]
  \begin{minipage}[t]{0.48\textwidth}
    \vspace{0pt}
    \begin{bchart}[max=9, width=0.85\textwidth]
      \bcbar[text=5, value=Yes, color=orange]{5}
      \bcbar[text=4, value=No, color=Aqua]{4}
    \end{bchart}
  \end{minipage}
  \begin{minipage}[t]{0.48\textwidth}
    \vspace{0pt}
    \begin{yamlcode}
      { Austria: Vienna,
        South Africa: {
          Executive: Pretoria,
          Judicial: Bloemfontein,
          Legislative: Cape Town}
      }
    \end{yamlcode}
  \end{minipage}
  \caption{Flow Style}
\end{figure}

\begin{figure}[H]
  \begin{minipage}[t]{0.48\textwidth}
    \vspace{0pt}
    \begin{bchart}[max=9, width=0.85\textwidth]
      \bcbar[text=7, value=Yes, color=orange]{7}
      \bcbar[text=2, value=No, color=Aqua]{2}
    \end{bchart}
  \end{minipage}
  \begin{minipage}[t]{0.48\textwidth}
    \vspace{0pt}
    \begin{yamlcode}
      Austria: Vienna
      South Africa:
        Executive:   Pretoria
        Judicial:    Bloemfontein
        Legislative: Cape Town
    \end{yamlcode}
  \end{minipage}
  \caption{Block Style}
\end{figure}

\begin{figure}[H]
  \begin{minipage}[t]{0.48\textwidth}
    \vspace{0pt}
    \begin{bchart}[max=9, width=0.85\textwidth]
      \bcbar[text=0, value=$\quad$Yes, color=orange]{0}
      \bcbar[text=9, value=No, color=Aqua]{9}
    \end{bchart}
  \end{minipage}
  \begin{minipage}[t]{0.48\textwidth}
    \vspace{0pt}
    \begin{yamlcode}
      ?
      - { 'pretty': complex key }
      - - 😱
      - Still part of the key
      : value
    \end{yamlcode}
  \end{minipage}
  \caption{Support for Complex Keys}
\end{figure}

\subsubsection{Multiple Documents}

\begin{figure}[H]
  \begin{minipage}[t]{0.48\textwidth}
    \vspace{0pt}
    \begin{bchart}[max=9, width=0.85\textwidth]
      \bcbar[text=0, value=$\quad$Yes, color=orange]{0}
      \bcbar[text=9, value=No, color=Aqua]{9}
    \end{bchart}
  \end{minipage}
  \begin{minipage}[t]{0.48\textwidth}
    \vspace{0pt}
    \begin{yamlcode}
      "Hello First Document"
      ...
      'Second Document'
      ...
      Third Document
    \end{yamlcode}
  \end{minipage}
  \caption{Support Streams}
\end{figure}

\subsubsection{Types}

\paragraph{Directives}

\begin{figure}[H]
  \begin{minipage}[t]{0.48\textwidth}
    \vspace{0pt}
    \begin{bchart}[max=9, width=0.85\textwidth]
      \bcbar[text=1, value=Yes, color=orange]{1}
      \bcbar[text=7, value=No, color=Aqua]{7}
      \bcbar[text=1, value=?, color=DarkTurquoise]{1}
    \end{bchart}
  \end{minipage}
  \begin{minipage}[t]{0.48\textwidth}
    \vspace{0pt}
    \begin{yamlcode}
      %YAML 1.2
    \end{yamlcode}
  \end{minipage}
  \caption{YAML Version}
\end{figure}

\begin{figure}[H]
  \begin{minipage}[t]{0.48\textwidth}
    \vspace{0pt}
    \begin{bchart}[max=9, width=0.85\textwidth]
      \bcbar[text=3, value=Yes, color=orange]{3}
      \bcbar[text=5, value=No, color=Aqua]{5}
      \bcbar[text=1, value=?, color=DarkTurquoise]{1}
    \end{bchart}
  \end{minipage}
  \begin{minipage}[t]{0.48\textwidth}
    \vspace{0pt}
    \begin{yamlcode}
      %TAG !       tag:yaml.org,2002:
      %TAG !!      tag:yaml.org,2002:
      %TAG !name! tag:yaml.org,2002:
      ---
    \end{yamlcode}
  \end{minipage}
  \caption{Tag Handle Definition}
\end{figure}

\begin{figure}[H]
  \begin{minipage}[t]{0.48\textwidth}
    \vspace{0pt}
    \begin{bchart}[max=9, width=0.85\textwidth]
      \bcbar[text=2, value=Yes, color=orange]{2}
      \bcbar[text=6, value=No, color=Aqua]{6}
      \bcbar[text=1, value=?, color=DarkTurquoise]{1}
    \end{bchart}
  \end{minipage}
  \begin{minipage}[t]{0.48\textwidth}
    \vspace{0pt}
    \begin{yamlcode}
      %TAG !name! tag:yaml.org,2002:
      ---
      !name!str 6 # "6"
    \end{yamlcode}
  \end{minipage}
  \caption{Named Tag Handle}
\end{figure}

\paragraph{Tags}

\subparagraph{Tag Shorthands}

\begin{figure}[H]
  \begin{minipage}[t]{0.48\textwidth}
    \vspace{0pt}
    \begin{bchart}[max=9, width=0.85\textwidth]
      \bcbar[text=4, value=Yes, color=orange]{4}
      \bcbar[text=4, value=No, color=Aqua]{4}
      \bcbar[text=1, value=?, color=DarkTurquoise]{1}
      \bcxlabel{}
    \end{bchart}
  \end{minipage}
  \begin{minipage}[t]{0.48\textwidth}
    \vspace{0pt}
    \begin{yamlcode}
      !suffix value
    \end{yamlcode}
  \end{minipage}
  \caption{Primary Tag Handle}
\end{figure}

\begin{figure}[H]
  \begin{minipage}[t]{0.48\textwidth}
    \vspace{0pt}
    \begin{bchart}[max=9, width=0.85\textwidth]
      \bcbar[text=3, value=Yes, color=orange]{3}
      \bcbar[text=5, value=No, color=Aqua]{5}
      \bcbar[text=1, value=?, color=DarkTurquoise]{1}
    \end{bchart}
  \end{minipage}
  \begin{minipage}[t]{0.48\textwidth}
    \vspace{0pt}
    \begin{yamlcode}
      !!suffix value
    \end{yamlcode}
  \end{minipage}
  \caption{Secondary Tag Handle}
\end{figure}

\begin{figure}[H]
\subparagraph{Verbatim Tags}
  \begin{minipage}[t]{0.48\textwidth}
    \vspace{0pt}
    \begin{bchart}[max=9, width=0.85\textwidth]
      \bcbar[text=0, value=$\quad$Yes, color=orange]{0}
      \bcbar[text=8, value=No, color=Aqua]{8}
      \bcbar[text=1, value=?, color=DarkTurquoise]{1}
    \end{bchart}
  \end{minipage}
  \begin{minipage}[t]{0.48\textwidth}
    \vspace{0pt}
    \begin{yamlcode}
      !<!ruby/object:Set> value
    \end{yamlcode}
  \end{minipage}
  \caption{Local Verbatim Tags}
\end{figure}

\begin{figure}[H]
  \begin{minipage}[t]{0.48\textwidth}
    \vspace{0pt}
    \begin{bchart}[max=9, width=0.85\textwidth]
      \bcbar[text=0, value=$\quad$Yes, color=orange]{0}
      \bcbar[text=8, value=No, color=Aqua]{8}
      \bcbar[text=1, value=?, color=DarkTurquoise]{1}
    \end{bchart}
  \end{minipage}
  \begin{minipage}[t]{0.48\textwidth}
    \vspace{0pt}
    \begin{yamlcode}
      !<tag:yaml.org,2002:str> value
    \end{yamlcode}
  \end{minipage}
  \caption{Global Verbatim Tags}
\end{figure}

\subparagraph{Other Tags}

\begin{figure}[H]
  \begin{minipage}[t]{0.48\textwidth}
    \vspace{0pt}
    \begin{bchart}[max=9, width=0.85\textwidth]
      \bcbar[text=0, value=$\quad$Yes, color=orange]{0}
      \bcbar[text=8, value=No, color=Aqua]{8}
      \bcbar[text=1, value=?, color=DarkTurquoise]{1}
    \end{bchart}
  \end{minipage}
  \begin{minipage}[t]{0.48\textwidth}
    \vspace{0pt}
    \begin{yamlcode}
      ! value
    \end{yamlcode}
  \end{minipage}
  \caption{Non-Specific Tag}
\end{figure}

\paragraph{Schemas}

\textbf{Remark:} One participant checked the box for the core schema without ticking the boxes for the failsafe and JSON schema. Since the core schema is an extended superset of the other two schemas, we counted the participants answers as a “Yes” vote for the failsafe and JSON schema.

\begin{figure}[H]
  \begin{minipage}[t]{0.48\textwidth}
    \vspace{0pt}
    \begin{bchart}[max=9, width=0.85\textwidth]
      \bcbar[text=5, value=Yes, color=orange]{5}
      \bcbar[text=3, value=No, color=Aqua]{3}
      \bcbar[text=1, value=?, color=DarkTurquoise]{1}
      \bcxlabel{}
    \end{bchart}
  \end{minipage}
  \begin{minipage}[t]{0.48\textwidth}
    \vspace{0pt}
    \begin{itemize}
      \item String
      \item Sequence
      \item Map
    \end{itemize}
  \end{minipage}
  \caption{Failsafe Schema}
\end{figure}

\begin{figure}[H]
  \begin{minipage}[t]{0.48\textwidth}
    \vspace{0pt}
    \begin{bchart}[max=9, width=0.85\textwidth]
      \bcbar[text=5, value=Yes, color=orange]{5}
      \bcbar[text=3, value=No, color=Aqua]{3}
      \bcbar[text=1, value=?, color=DarkTurquoise]{1}
    \end{bchart}
  \end{minipage}
  \begin{minipage}[t]{0.48\textwidth}
    \vspace{0pt}
    Failsafe Schema + JSON Types:
    \begin{minipage}[t]{2cm}
      \begin{itemize}[leftmargin=*]
        \item Null
        \item Boolean
        \item Integer
        \item Float
      \end{itemize}
    \end{minipage}
  \end{minipage}
  \caption{JSON Schema}
\end{figure}

\begin{figure}[H]
  \begin{minipage}[t]{0.48\textwidth}
    \vspace{0pt}
    \begin{bchart}[max=9, width=0.85\textwidth]
      \bcbar[text=3, value=Yes, color=orange]{3}
      \bcbar[text=5, value=No, color=Aqua]{5}
      \bcbar[text=1, value=?, color=DarkTurquoise]{1}
    \end{bchart}
  \end{minipage}
  \begin{minipage}[t]{0.48\textwidth}
    \vspace{0pt}
    JSON Schema and
      \vspace{-0.5cm}
      \begin{itemize}
        \item Octal/Hex: \code{0o123}, \code{0xfefe}
        \item Multiple Notations for same value:
              \code{null}, \code{Null}, \code{~}
      \end{itemize}
  \end{minipage}
  \caption{Core Schema}
\end{figure}

\begin{figure}[H]
  \begin{minipage}[t]{0.48\textwidth}
    \vspace{0pt}
    \begin{bchart}[max=9, width=0.85\textwidth]
      \bcbar[text=3, value=Yes, color=orange]{3}
      \bcbar[text=5, value=No, color=Aqua]{5}
      \bcbar[text=1, value=?, color=DarkTurquoise]{1}
    \end{bchart}
  \end{minipage}
  \begin{minipage}[t]{0.48\textwidth}
    \vspace{0pt}
    \begin{itemize}
      \item Ordered Map
      \item Set
      \item Binary
      \item Time
      \item …
    \end{itemize}
  \end{minipage}
  \caption{Additional Types}
\end{figure}

\subparagraph{Which Additional Types:}
\begin{itemize}
  \item “” (No answer)
  \item “binary”
  \item “date (but implemented in plugins)”
\end{itemize}

\subsubsection{References}

\begin{figure}[H]
  \begin{minipage}[t]{0.48\textwidth}
    \vspace{0pt}
    \begin{bchart}[max=9, width=0.85\textwidth]
      \bcbar[text=7, value=Yes, color=orange]{7}
      \bcbar[text=2, value=No, color=Aqua]{2}
    \end{bchart}
  \end{minipage}
  \begin{minipage}[t]{0.48\textwidth}
    \vspace{0pt}
    \begin{yamlcode}
      flowers: &flowers
        🌳🌸🌼
      garden:
        - *flowers # 🌳🌸🌼
        - *flowers # 🌳🌸🌼
    \end{yamlcode}
  \end{minipage}
  \caption{Support Anchors \& Aliases}
\end{figure}

\subsection{Interpretation}

The results of the survey showed that the participants preferred double quoted flow scalars over single quoted and plain scalars. A reasons for this could be that those scalars are familiar from other languages such as C, and that they are able to express arbitrary data. Asked about block scalar styles most of the Elektra developers did not think that any of the two styles were necessary.

In contrast to the decision about block scalars the participants preferred the block styles of sequences and mappings (collections) over the respective flow style. However, they also decided that a minimally useful YAML subset should include flow collections.

The Elektra developers decided against most of the specialized type features of YAML. Only the result count for and against primary tag handles resulted in a draw. In this case we decided to implement the feature, since it allows us to explicitly specify application specific types.

The questions about general type support (schemas) showed that a minimal YAML subset should include all types of the JSON Schema.

One of the few specialized features deemed necessary by the participants were anchors and aliases. These two elements allow can be used to reference the same data multiple times in the same document.

\subsubsection{Summary}

According to the results of the discussion we decided to implement the following YAML features for a minimal YAML parser:

\begin{itemize}
  \item Double Quoted Flow Scalars
  \item Block and Flow Collections
  \item JSON Schema
  \item Primary Tag Handle
  \item References
\end{itemize}
