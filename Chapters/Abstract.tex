\begin{abstract}
% Context: Elektra
Almost any application uses configuration settings to adapt its behavior to specific contexts. In ever growing complex computer systems, the large amount of different settings for different software can easily lead to misconfiguration. The different methods to configure systems and the large amount of different configuration file formats does certainly not improve the situation. Elektra is a plugin-based configuration framework that integrates configuration settings into a global hierarchical database and provides a common interface for configuration settings to make misconfiguration less likely.

% Problem: Configuration File Parsing
For Elektra to work without, or with only minor modifications to an application it needs to understand the configuration file format of said application. To do that Elektra uses a plugin-based system to parse configuration files and convert them into its native data structures. Changes to this data structure are then written back, effectively changing the configuration of an application via Elektra’s interface. Since there exist a plethora of different configuration file formats we need to think how we can parse them with as little effort as possible.

% Method
To evaluate which parsing system offers the best fit for the task of configuration file parsing, we first studied promising techniques in a detailed literature research. As example syntax for further analysis we used a subset of the popular language YAML that we determined using data from a survey we conducted with some of Elektra’s developers. Afterwards we wrote, generated and integrated parsing code, creating 8 new Elektra plugins, and extending one of Elektra’s existing plugins in the process. We then compared five of our parser plugins in a detailed evaluation.

% Results/Findings
The comparison of the various features of the parsing systems showed that for our example syntax \glstext{ANTLR} provided the most complete feature set and good error messages without requiring any changes to the grammar. While the parser generated by \glstext{ANTLR} was neither the fastest parser, nor the parser with the lowest memory overhead, the benchmarks in our comparison still show acceptable performance for our example data. To convert 10 000 lines of YAML the parser needed roughly 120ms on our test system.
\end{abstract}
