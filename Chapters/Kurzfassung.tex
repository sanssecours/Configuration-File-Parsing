\begin{kurzfassung}
\begin{sloppypar}
Nahezu jede Applikation verwendet Konfigurationseinstellungen um ihre Verhalten einem spezifischen Kontext anzupassen. In immer komplexer werdenden Computersystemen, kann die große Anzahl verschiedener Einstellungen leicht zu Fehlkonfigurationen führen. Die verschiedenen Möglichkeiten Softwaresysteme zu konfigurieren und die große Anzahl verschiedener Konfigurationsformate verbessert diese Situation sicherlich nicht. Elektra ist ein auf Plugin basierendes Konfigurationsframework, das Konfigurationseinstellungen in eine globale hierarchische Datenbank integriert und eine einheitliche Schnittstelle für diese anbietet um Fehlkonfigurationen zu vermeiden.
\end{sloppypar}

Damit Elektra ohne oder nur mit geringen Modifikationen an einer Applikation arbeiten kann, muss es das Konfigurationsformat der Applikation verstehen. Um das zu bewerkstelligen verwendet Elektra ein auf Plugins basierendes System um Konfigurationsdateien in seine internen Datenstrukturen umzuwandeln. Veränderungen der internen Datenstrukturen werden dann zurückgeschrieben, um so die Konfiguration einer Applikation über Elektras Schnittstelle zu ändern. Nachdem eine Vielzahl von verschiedenen Konfigurationsformaten existiert, müssen wir uns Gedanken machen, wie wir diese mit möglichst geringem Aufwand syntaktisch analysieren, also parsen, können.

\begin{sloppypar}
Um zu untersuchen, welches Syntaxanalysesystem die beste Lösung für die Aufgabe des Parsens von Konfigurationsdateien liefert analysieren wir zunächst  erfolgversprechende Parsing-Techniken in einer Literaturrecherche. Als Beispielsyntax für weitere Untersuchungen verwendeten wir ein Subset der populären Sprache YAML, das wir mit Hilfe der Daten einer Befragung von einigen Elektra-Entwicklern ermittelten. Danach schrieben, generierten und integrierten wir Parsing-Code für 8 neue Elektra-Plugins und erweiterten eines der existierenden Plugins. Schließlich verglichen wir fünf der Parser-Plugins in einer detaillierten Untersuchung.
\end{sloppypar}

Der Vergleich der Features der Parsersysteme zeigt, dass für unsere Beispielsprache \glstext{ANTLR} den größten Funktionsumfang und gute Fehlermeldungen liefert, ohne Grammatikänderungen zu benötigen. Während der Code des Generators weder die schnellste Laufzeit, noch den geringsten Speicherverbrauch aufweist, zeigen unsere Messungen eine angemessene Leistung. Um 10 000 Zeilen YAML zu konvertieren benötigt der Parser ungefähr 120ms auf unserem Testsystem.
\end{kurzfassung}
