\chapter{Conclusion}

In the thesis we compared different parsing techniques using the \nameref{sec:keyset} structures of the configuration framework \href{https://www.libelektra.org}{Elektra} as common end product of the parsing process. The aim of this task was to find the most promising technique for a \glstext{YAML} storage plugin for Elektra.

We first looked at the current state of literature to determine the most promising parsing techniques. One of the results of this task was that we only considered directed parsing algorithms. They are powerful enough to parse \glstext{YAML} and offer better runtime performance than the more powerful parsing techniques that parse the input in arbitrary order.

Afterwards we looked for libraries and parser generators that implemented the chosen parsing algorithms: recursive descent, \gls{ALL(*)}, LR, Earley, \gls{PEG} parser and combinatory parsing. We decided to only consider C and C++ as implementation languages to improve the comparability of the parsing code. Since Elektra also offers support for the parsing library Augeas we decided to also add this library to our list of implementation options. At the end of this process our list of parser library and parser generators consisted of

\begin{itemize}
  \item handwritten code (recursive descent),
  \item \href{http://www.antlr.org}{ANTLR} (\gls{ALL(*)}),
  \item \href{https://www.gnu.org/software/bison}{Bison} (LR),
  \item \href{https://github.com/vnmakarov/yaep}{YAEP} (Earley parser),
  \item \href{https://github.com/taocpp/PEGTL}{PEGTL} (PEG parser),
  \item \href{https://github.com/orangeduck/mpc}{mpc} (parser combinator), and
  \item \href{http://augeas.net}{Augeas} (bidirectional programming)
\end{itemize}

. As the specification of \glstext{YAML} is a rather extensive we decided to only implement a subset of the language. We discussed the specific required features for such a subset with Elektra’s developers as part of a presentation about the features of the language. The questionnaire we

% Found minimal YAML subset in discussion with Elektra developers:
% → Developers preferred double quoted scalars
% → Did not implement all of the stated features like typing and references, since they are either not really applicable to Elektra (references) or require work that is out of the scope of the thesis (typing)

% Determined mapping between YAML and Elektra `keySet`:
% → Requires way to translate non-leaf values to leaf values

% Implemented plugins:
% → Problems
%   → Begin and end markers in lexer based plugins
%   → Handling dynamic context in PEG parser
%   → Augeas not powerful enough to handle YAML

% Comparison:
% → All parsing engines show linear runtime/memory usage for tested file
% → Surprisingly good performance results of Earley parser
% → Sometimes large difference between operating systems
% → ANTLR provides most support code, other plugins required reimplementation of the support code
% → ANTLR provides good error messages without requiring any changes to the grammar
% → ANTLR most promising because of:
%   → support code
%   → good error messages
%   → large community
