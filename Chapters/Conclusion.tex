\chapter{Conclusion}

% References
% ==========
%
% - https://patthomson.net/2018/01/29/concluding-the-thesis-four-key-actions
%
% Guidelines
% ==========
%
% - To what extent you achieved your aims/objectives OR not: if not, why not?
% - How important and significant your results are, as well as any limitations
%   of your research (e.g. small sample size; other variables)
% - Where the research should go from here: what are some interesting further
%   areas to be explored based on what you have discovered or proven?
%
% - restates the question
% - provides a succinct summary of the answer(s) and how this was produced ( I
%   did this and my analysis showed 1, 2, 3 and I argue that this… ). The writer
%   usually acknowledges the particularity of the research here too (sometimes
%   called limitations.)
% - shows how the research contributes to the literatures (the contribution of
%   the research is a, b, c)
% - discusses the implications (the results could lead to further research on,
%   changes in policy/practice such as.. ). The implications arise logically from
%   the particularity of the study and its results – they point to questions the
%   study opens up, what the results says to current thinking about and acting on
%   the topic.
%
% Contribution
% ============
%
% - Real world comparison of common file parsing techniques
% - Common data structure (better comparison possibilities)
%
% Limitations
% ===========
%
% - Only looked at YAML: Other configuration file formats would be interesting too
% - Additional parsing libraries in other programming languages might be better
%   for parsing config data
% - Did not use Augeas (not powerful enough) and mpc (less powerful version of
%   PEGTL)
% - More test data for performance test (parser more complete set of YAML)
%
% Future Work
% ===========
%
% - Add lexer level error messages
% - Add escaping and unescaping support for single and double quoted scalars

In the thesis we compared different parsing techniques using the \nameref{sec:keyset} structures of the configuration framework \href{https://www.libelektra.org}{Elektra} as common end product of the parsing process. The aim of this task was to find the most promising parsing technique for configuration files using the language \glstext{YAML} as example.
