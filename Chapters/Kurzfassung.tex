\begin{kurzfassung}
Parsen – das Erfassen der Struktur eines Textes an Hand einer Grammatik – ist ein wichtiger Teil der Informatikforschung. Man würde vermuten dass die Resultate dieser Forschung, Parsergeneratoren und Parserbibliotheken, vorherrschende Werkzeuge sind, wenn Programmierer eine Computersprache verarbeiten. Das ist jedoch nicht der Fall. Insbesondere für spezialisierte und einfache Sprachen, wie sie z.B. für Konfigurationsdaten verwendet werden, entscheiden sich Programmierer oft ihren eigenen Parser zu schreiben und ignorieren dabei in der Vergangenheit gewonnene Erkenntnisse. So gibt es viele Funktionen wie z.B. Unicode-Support und Fehlerbehebung, die ein Entwickler normalerweise nicht in seinen handgeschriebenen Code integrieren wird. Die Verwendung einer Bibliothek für eine Computersprache kann diese Problemen lösen. Allerdings existieren solche Bibliotheken nicht für jede Sprache oder bieten keinen Zugriff auf wichtige Daten eines Textes, wie z.B. Kommentare. Um die gegenwärtige Situation zu verbessern beschäftigt sich diese Diplomarbeit mit aktuellen Parserwerkzeugen und vergleicht diese. Für diese Zweck parsen wir eine Teilmenge der Sprache \glstext{YAML} und wandeln Dokumente dieser Sprache in die Datenstrukturen des Konfiguration-Frameworks Elektra um. Die gemeinsame Datenstrukturen erlauben einen sachlichen Vergleich der Parser die üblicherweise verschiedene Arten von Syntax-Bäumen erzeugen oder benutzerdefinierten Code während des Parsens ausführen.

Diese Arbeit beginnt mit einem Literaturrecherche über den aktuellen Stand der Forschung im Bereich Parsing. Nach dieser Recherche wählen wir Werkzeuge für die vielversprechendsten Pasertechniken, \glstext{ALL(*)}, LR, Earley Parser, \glstext{PEG}, Paserkombinatoren und bidirektionale Programmierung. In einer Diskussion mit Elektra-Entwickler entschieden wir uns dann für die Funktionen, die unsere \glstext{YAML}-Teilmenge unterstützen soll. Schließlich gehen wir auf die Abbildung zwischen YAML und Elektras Datenstrukturen ein, und implementieren die Parser für die \glstext{YAML}-Teilmenge. In dieser Phase entschieden wir uns dabei gegen eine Implementierung mit der Paserkombinatorbibliothek mpc und der Bibliothek für bidirektionale Programmierung Augeas. Während Augeas nicht mächtig genug ist um die kontextsensitive Sprache \glstext{YAML} zu verarbeiten, konnten wir keinen Vorteil von mpc gegenüber der konzeptuell ähnlichen \glstext{PEG}-Bibliothek \glstext{PEGTL} finden.

Der Vergleich der Parser zeigt, dass für unser Beispielsprache \glstext{ANTLR} den größten Funktionsumfang, und gute Fehlermeldungen liefert, ohne Grammatikänderungen zu benötigen. Während der Code des Generators weder die schnellste Laufzeit, noch den geringsten Speicherverbrauch aufweist, zeigen unsere Messungen des Codes eine angemessene Leistung.
\end{kurzfassung}
